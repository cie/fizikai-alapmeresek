\documentclass[12pt]{article}
\usepackage[utf8]{inputenc}
\usepackage[final]{graphics}
\usepackage[pdftex]{graphicx}
\usepackage[dvips]{color}
\usepackage{amsfonts}
\usepackage{subfigure}
\usepackage{lscape}
\usepackage{hyperref}
\usepackage{amsmath}
\usepackage{units}
\usepackage{float}
\usepackage[table]{xcolor}
\usepackage{rmpage}
\usepackage[magyar]{babel}
\usepackage[margin=2cm]{geometry}
\newcommand{\hide}[1]{}
\DeclareGraphicsExtensions{.jpg,.pdf,.mps,.png}
    \title{Optikai alapmérések}
    \author{Kalló Bernát -- Mérés: 2012.05.09. -- Leadás: 2012.06.08.}
    \date{}
    \begin{document}
\def\tnagy{\ensuremath{t_{\text{nagy}}}}\def\tkicsi{\ensuremath{t_{\text{kicsi}}}}\def\xnagy{\ensuremath{x_{\text{nagy}}}}\def\xkicsi{\ensuremath{x_{\text{kicsi}}}}\def\gnagy{\ensuremath{g_{\text{nagy}}}}\def\gkicsi{\ensuremath{g_{\text{kicsi}}}}\def\Deltagnagy{\ensuremath{\Delta{}g_{\text{nagy}}}}\def\Deltagkicsi{\ensuremath{\Delta{}g_{\text{kicsi}}}}\def\geff{\ensuremath{g_\text{eff}}}

\maketitle

\section{Törésmutató meghatározása}

Két közeg határán áthaladó fénysugár megtörik, úgy, hogy a beesési és törési szögek szinuszának aránya \[\frac{\sin\alpha}{\sin\beta} = n_{2,1}\] állandó, a két közeg relatív törésmutatója.

Megmértük egy félkör alakú lencse egyenes oldalán belépő fénysugár törési szögét. Valójában a kilépő fénysugár szögét mértük, de mivel a görbe oldalon való kilépéskor a sugár merőleges a határfelületre, az iránya ott nem változik meg. A mért adatok az alábbi táblázatban láthatók:

  
  \begin{table}[H]
    \begin{center}
      \begin{tabular}{|
c|
c|
}
        \hline
        
\ensuremath{\unit[\text{\ensuremath{\alpha}}]{({}^\circ)}} & 
\ensuremath{\unit[\text{\ensuremath{\beta}}]{({}^\circ)}}
\\
        \hline\hline
        
0
 & 0
\\
        \hline
        
10
 & 7
\\
        \hline
        
20
 & 13
\\
        \hline
        
30
 & 20
\\
        \hline
        
40
 & 26
\\
        \hline
        
50
 & 31
\\
        \hline
        
60
 & 36
\\
        \hline
        
70
 & 39
\\
        \hline
        
80
 & 42
\\
        \hline
      \end{tabular}
      \caption{Beesési és törési szögek belépéskor}
      \label{tab:}
    \end{center}
  \end{table}

Ezt grafikonon ábrázolva:

  \begin{figure}[H]
    \begin{center}
% GNUPLOT: LaTeX picture

\setlength{\unitlength}{0.240900pt}

\ifx\plotpoint\undefined\newsavebox{\plotpoint}\fi

\sbox{\plotpoint}{\rule[-0.200pt]{0.400pt}{0.400pt}}%

\begin{picture}(1653,1653)(0,0)

\sbox{\plotpoint}{\rule[-0.200pt]{0.400pt}{0.400pt}}%

\put(211.0,131.0){\rule[-0.200pt]{4.818pt}{0.400pt}}

\put(191,131){\makebox(0,0)[r]{ 0}}

\put(1583.0,131.0){\rule[-0.200pt]{4.818pt}{0.400pt}}

\put(211.0,427.0){\rule[-0.200pt]{4.818pt}{0.400pt}}

\put(191,427){\makebox(0,0)[r]{ 0.2}}

\put(1583.0,427.0){\rule[-0.200pt]{4.818pt}{0.400pt}}

\put(211.0,724.0){\rule[-0.200pt]{4.818pt}{0.400pt}}

\put(191,724){\makebox(0,0)[r]{ 0.4}}

\put(1583.0,724.0){\rule[-0.200pt]{4.818pt}{0.400pt}}

\put(211.0,1020.0){\rule[-0.200pt]{4.818pt}{0.400pt}}

\put(191,1020){\makebox(0,0)[r]{ 0.6}}

\put(1583.0,1020.0){\rule[-0.200pt]{4.818pt}{0.400pt}}

\put(211.0,1317.0){\rule[-0.200pt]{4.818pt}{0.400pt}}

\put(191,1317){\makebox(0,0)[r]{ 0.8}}

\put(1583.0,1317.0){\rule[-0.200pt]{4.818pt}{0.400pt}}

\put(211.0,1613.0){\rule[-0.200pt]{4.818pt}{0.400pt}}

\put(191,1613){\makebox(0,0)[r]{ 1}}

\put(1583.0,1613.0){\rule[-0.200pt]{4.818pt}{0.400pt}}

\put(211.0,131.0){\rule[-0.200pt]{0.400pt}{4.818pt}}

\put(211,90){\makebox(0,0){ 0}}

\put(211.0,1593.0){\rule[-0.200pt]{0.400pt}{4.818pt}}

\put(489.0,131.0){\rule[-0.200pt]{0.400pt}{4.818pt}}

\put(489,90){\makebox(0,0){ 0.2}}

\put(489.0,1593.0){\rule[-0.200pt]{0.400pt}{4.818pt}}

\put(768.0,131.0){\rule[-0.200pt]{0.400pt}{4.818pt}}

\put(768,90){\makebox(0,0){ 0.4}}

\put(768.0,1593.0){\rule[-0.200pt]{0.400pt}{4.818pt}}

\put(1046.0,131.0){\rule[-0.200pt]{0.400pt}{4.818pt}}

\put(1046,90){\makebox(0,0){ 0.6}}

\put(1046.0,1593.0){\rule[-0.200pt]{0.400pt}{4.818pt}}

\put(1325.0,131.0){\rule[-0.200pt]{0.400pt}{4.818pt}}

\put(1325,90){\makebox(0,0){ 0.8}}

\put(1325.0,1593.0){\rule[-0.200pt]{0.400pt}{4.818pt}}

\put(1603.0,131.0){\rule[-0.200pt]{0.400pt}{4.818pt}}

\put(1603,90){\makebox(0,0){ 1}}

\put(1603.0,1593.0){\rule[-0.200pt]{0.400pt}{4.818pt}}

\put(211.0,131.0){\rule[-0.200pt]{0.400pt}{357.014pt}}

\put(211.0,131.0){\rule[-0.200pt]{335.333pt}{0.400pt}}

\put(1603.0,131.0){\rule[-0.200pt]{0.400pt}{357.014pt}}

\put(211.0,1613.0){\rule[-0.200pt]{335.333pt}{0.400pt}}

\put(70,872){\makebox(0,0){$\sin\alpha$}}

\put(907,29){\makebox(0,0){$\sin\beta$}}

\put(211,131){\raisebox{-.8pt}{\makebox(0,0){$\Diamond$}}}

\put(381,388){\raisebox{-.8pt}{\makebox(0,0){$\Diamond$}}}

\put(524,638){\raisebox{-.8pt}{\makebox(0,0){$\Diamond$}}}

\put(687,872){\raisebox{-.8pt}{\makebox(0,0){$\Diamond$}}}

\put(821,1084){\raisebox{-.8pt}{\makebox(0,0){$\Diamond$}}}

\put(928,1266){\raisebox{-.8pt}{\makebox(0,0){$\Diamond$}}}

\put(1029,1414){\raisebox{-.8pt}{\makebox(0,0){$\Diamond$}}}

\put(1087,1524){\raisebox{-.8pt}{\makebox(0,0){$\Diamond$}}}

\put(1142,1590){\raisebox{-.8pt}{\makebox(0,0){$\Diamond$}}}

\multiput(212,131)(11.312,17.402){2}{\usebox{\plotpoint}}

\put(234.18,166.08){\usebox{\plotpoint}}

\put(245.17,183.69){\usebox{\plotpoint}}

\put(256.31,201.20){\usebox{\plotpoint}}

\multiput(267,218)(11.143,17.511){2}{\usebox{\plotpoint}}

\put(289.74,253.73){\usebox{\plotpoint}}

\put(300.88,271.24){\usebox{\plotpoint}}

\multiput(309,284)(10.792,17.729){2}{\usebox{\plotpoint}}

\put(334.39,323.71){\usebox{\plotpoint}}

\put(345.70,341.11){\usebox{\plotpoint}}

\put(356.85,358.62){\usebox{\plotpoint}}

\multiput(366,373)(11.143,17.511){2}{\usebox{\plotpoint}}

\put(390.28,411.15){\usebox{\plotpoint}}

\put(401.19,428.80){\usebox{\plotpoint}}

\put(412.11,446.45){\usebox{\plotpoint}}

\multiput(422,462)(11.143,17.511){2}{\usebox{\plotpoint}}

\put(445.54,498.99){\usebox{\plotpoint}}

\put(456.68,516.50){\usebox{\plotpoint}}

\put(467.82,534.01){\usebox{\plotpoint}}

\multiput(478,550)(10.792,17.729){2}{\usebox{\plotpoint}}

\put(500.80,586.82){\usebox{\plotpoint}}

\put(511.94,604.33){\usebox{\plotpoint}}

\put(523.08,621.84){\usebox{\plotpoint}}

\multiput(534,639)(11.143,17.511){2}{\usebox{\plotpoint}}

\put(556.93,674.10){\usebox{\plotpoint}}

\put(568.19,691.53){\usebox{\plotpoint}}

\multiput(577,706)(11.143,17.511){2}{\usebox{\plotpoint}}

\put(601.33,744.24){\usebox{\plotpoint}}

\put(612.48,761.75){\usebox{\plotpoint}}

\put(623.62,779.26){\usebox{\plotpoint}}

\multiput(633,794)(10.792,17.729){2}{\usebox{\plotpoint}}

\put(656.59,832.07){\usebox{\plotpoint}}

\put(667.74,849.58){\usebox{\plotpoint}}

\put(678.88,867.10){\usebox{\plotpoint}}

\multiput(689,883)(11.143,17.511){2}{\usebox{\plotpoint}}

\put(712.31,919.63){\usebox{\plotpoint}}

\put(723.25,937.26){\usebox{\plotpoint}}

\put(734.14,954.93){\usebox{\plotpoint}}

\multiput(745,972)(11.143,17.511){2}{\usebox{\plotpoint}}

\put(767.57,1007.46){\usebox{\plotpoint}}

\put(778.71,1024.97){\usebox{\plotpoint}}

\multiput(787,1038)(11.692,17.149){2}{\usebox{\plotpoint}}

\put(812.50,1077.26){\usebox{\plotpoint}}

\put(823.53,1094.84){\usebox{\plotpoint}}

\put(834.68,1112.35){\usebox{\plotpoint}}

\multiput(844,1127)(11.143,17.511){2}{\usebox{\plotpoint}}

\put(868.11,1164.88){\usebox{\plotpoint}}

\put(879.25,1182.39){\usebox{\plotpoint}}

\put(890.25,1199.99){\usebox{\plotpoint}}

\multiput(900,1216)(11.143,17.511){2}{\usebox{\plotpoint}}

\put(923.36,1252.72){\usebox{\plotpoint}}

\put(934.51,1270.23){\usebox{\plotpoint}}

\put(945.65,1287.74){\usebox{\plotpoint}}

\multiput(956,1304)(11.143,17.511){2}{\usebox{\plotpoint}}

\put(978.79,1340.45){\usebox{\plotpoint}}

\put(989.77,1358.06){\usebox{\plotpoint}}

\put(1000.91,1375.57){\usebox{\plotpoint}}

\multiput(1012,1393)(11.692,17.149){2}{\usebox{\plotpoint}}

\put(1035.04,1427.64){\usebox{\plotpoint}}

\put(1046.19,1445.15){\usebox{\plotpoint}}

\multiput(1055,1459)(10.792,17.729){2}{\usebox{\plotpoint}}

\put(1079.16,1497.97){\usebox{\plotpoint}}

\put(1090.30,1515.48){\usebox{\plotpoint}}

\put(1101.45,1532.99){\usebox{\plotpoint}}

\multiput(1111,1548)(11.143,17.511){2}{\usebox{\plotpoint}}

\put(1134.88,1585.52){\usebox{\plotpoint}}

\put(1145.88,1603.12){\usebox{\plotpoint}}

\put(1152,1613){\usebox{\plotpoint}}

\put(211.0,131.0){\rule[-0.200pt]{0.400pt}{357.014pt}}

\put(211.0,131.0){\rule[-0.200pt]{335.333pt}{0.400pt}}

\put(1603.0,131.0){\rule[-0.200pt]{0.400pt}{357.014pt}}

\put(211.0,1613.0){\rule[-0.200pt]{335.333pt}{0.400pt}}

\end{picture}
    \end{center}
\caption{Beesési ($\alpha$) és törési ($\beta$) szögek belépéskor}  \end{figure}

Az illesztett egyenes meredeksége $n_{2,1} = \ensuremath{1,481}$, tehát ez a műanyagnak a levegőre vonatkoztatott törésmutatója.

Ezután elvégeztük a kísérletet fordítva is, a kilépéskor történő törést mérve.
  
  \begin{table}[H]
    \begin{center}
      \begin{tabular}{|
c|
c|
}
        \hline
        
\ensuremath{\unit[\text{\ensuremath{\alpha}}]{({}^\circ)}} & 
\ensuremath{\unit[\text{\ensuremath{\beta}}]{({}^\circ)}}
\\
        \hline\hline
        
0
 & 0
\\
        \hline
        
10
 & 14
\\
        \hline
        
15
 & 21
\\
        \hline
        
20
 & 29
\\
        \hline
        
25
 & 37
\\
        \hline
        
30
 & 46
\\
        \hline
        
35
 & 56
\\
        \hline
        
40
 & 69
\\
        \hline
      \end{tabular}
      \caption{Beesési és törési szögek kilépéskor}
      \label{tab:}
    \end{center}
  \end{table}

Grafikonon ábrázolva:

  \begin{figure}[H]
    \begin{center}
% GNUPLOT: LaTeX picture

\setlength{\unitlength}{0.240900pt}

\ifx\plotpoint\undefined\newsavebox{\plotpoint}\fi

\sbox{\plotpoint}{\rule[-0.200pt]{0.400pt}{0.400pt}}%

\begin{picture}(1653,1653)(0,0)

\sbox{\plotpoint}{\rule[-0.200pt]{0.400pt}{0.400pt}}%

\put(211.0,131.0){\rule[-0.200pt]{4.818pt}{0.400pt}}

\put(191,131){\makebox(0,0)[r]{ 0}}

\put(1583.0,131.0){\rule[-0.200pt]{4.818pt}{0.400pt}}

\put(211.0,427.0){\rule[-0.200pt]{4.818pt}{0.400pt}}

\put(191,427){\makebox(0,0)[r]{ 0.2}}

\put(1583.0,427.0){\rule[-0.200pt]{4.818pt}{0.400pt}}

\put(211.0,724.0){\rule[-0.200pt]{4.818pt}{0.400pt}}

\put(191,724){\makebox(0,0)[r]{ 0.4}}

\put(1583.0,724.0){\rule[-0.200pt]{4.818pt}{0.400pt}}

\put(211.0,1020.0){\rule[-0.200pt]{4.818pt}{0.400pt}}

\put(191,1020){\makebox(0,0)[r]{ 0.6}}

\put(1583.0,1020.0){\rule[-0.200pt]{4.818pt}{0.400pt}}

\put(211.0,1317.0){\rule[-0.200pt]{4.818pt}{0.400pt}}

\put(191,1317){\makebox(0,0)[r]{ 0.8}}

\put(1583.0,1317.0){\rule[-0.200pt]{4.818pt}{0.400pt}}

\put(211.0,1613.0){\rule[-0.200pt]{4.818pt}{0.400pt}}

\put(191,1613){\makebox(0,0)[r]{ 1}}

\put(1583.0,1613.0){\rule[-0.200pt]{4.818pt}{0.400pt}}

\put(211.0,131.0){\rule[-0.200pt]{0.400pt}{4.818pt}}

\put(211,90){\makebox(0,0){ 0}}

\put(211.0,1593.0){\rule[-0.200pt]{0.400pt}{4.818pt}}

\put(489.0,131.0){\rule[-0.200pt]{0.400pt}{4.818pt}}

\put(489,90){\makebox(0,0){ 0.2}}

\put(489.0,1593.0){\rule[-0.200pt]{0.400pt}{4.818pt}}

\put(768.0,131.0){\rule[-0.200pt]{0.400pt}{4.818pt}}

\put(768,90){\makebox(0,0){ 0.4}}

\put(768.0,1593.0){\rule[-0.200pt]{0.400pt}{4.818pt}}

\put(1046.0,131.0){\rule[-0.200pt]{0.400pt}{4.818pt}}

\put(1046,90){\makebox(0,0){ 0.6}}

\put(1046.0,1593.0){\rule[-0.200pt]{0.400pt}{4.818pt}}

\put(1325.0,131.0){\rule[-0.200pt]{0.400pt}{4.818pt}}

\put(1325,90){\makebox(0,0){ 0.8}}

\put(1325.0,1593.0){\rule[-0.200pt]{0.400pt}{4.818pt}}

\put(1603.0,131.0){\rule[-0.200pt]{0.400pt}{4.818pt}}

\put(1603,90){\makebox(0,0){ 1}}

\put(1603.0,1593.0){\rule[-0.200pt]{0.400pt}{4.818pt}}

\put(211.0,131.0){\rule[-0.200pt]{0.400pt}{357.014pt}}

\put(211.0,131.0){\rule[-0.200pt]{335.333pt}{0.400pt}}

\put(1603.0,131.0){\rule[-0.200pt]{0.400pt}{357.014pt}}

\put(211.0,1613.0){\rule[-0.200pt]{335.333pt}{0.400pt}}

\put(70,872){\makebox(0,0){$\sin\alpha$}}

\put(907,29){\makebox(0,0){$\sin\beta$}}

\put(211,131){\raisebox{-.8pt}{\makebox(0,0){$\Diamond$}}}

\put(548,388){\raisebox{-.8pt}{\makebox(0,0){$\Diamond$}}}

\put(710,515){\raisebox{-.8pt}{\makebox(0,0){$\Diamond$}}}

\put(886,638){\raisebox{-.8pt}{\makebox(0,0){$\Diamond$}}}

\put(1049,757){\raisebox{-.8pt}{\makebox(0,0){$\Diamond$}}}

\put(1212,872){\raisebox{-.8pt}{\makebox(0,0){$\Diamond$}}}

\put(1365,981){\raisebox{-.8pt}{\makebox(0,0){$\Diamond$}}}

\put(1511,1084){\raisebox{-.8pt}{\makebox(0,0){$\Diamond$}}}

\put(211,141){\usebox{\plotpoint}}

\put(211.00,141.00){\usebox{\plotpoint}}

\put(227.79,153.19){\usebox{\plotpoint}}

\put(244.29,165.78){\usebox{\plotpoint}}

\put(261.18,177.84){\usebox{\plotpoint}}

\put(278.07,189.91){\usebox{\plotpoint}}

\put(294.49,202.60){\usebox{\plotpoint}}

\put(311.36,214.69){\usebox{\plotpoint}}

\put(328.37,226.58){\usebox{\plotpoint}}

\put(345.22,238.67){\usebox{\plotpoint}}

\put(361.87,251.05){\usebox{\plotpoint}}

\put(378.76,263.11){\usebox{\plotpoint}}

\put(395.59,275.25){\usebox{\plotpoint}}

\put(412.05,287.89){\usebox{\plotpoint}}

\put(428.94,299.96){\usebox{\plotpoint}}

\put(445.50,312.46){\usebox{\plotpoint}}

\put(462.23,324.74){\usebox{\plotpoint}}

\put(479.12,336.80){\usebox{\plotpoint}}

\put(495.88,349.04){\usebox{\plotpoint}}

\put(512.41,361.58){\usebox{\plotpoint}}

\put(529.30,373.64){\usebox{\plotpoint}}

\put(546.19,385.71){\usebox{\plotpoint}}

\put(562.94,397.96){\usebox{\plotpoint}}

\put(579.83,410.02){\usebox{\plotpoint}}

\put(596.72,422.09){\usebox{\plotpoint}}

\put(613.32,434.54){\usebox{\plotpoint}}

\put(630.01,446.87){\usebox{\plotpoint}}

\put(646.90,458.93){\usebox{\plotpoint}}

\put(663.30,471.65){\usebox{\plotpoint}}

\put(680.19,483.71){\usebox{\plotpoint}}

\put(697.08,495.77){\usebox{\plotpoint}}

\put(713.60,508.33){\usebox{\plotpoint}}

\put(730.37,520.55){\usebox{\plotpoint}}

\put(747.26,532.62){\usebox{\plotpoint}}

\put(763.98,544.91){\usebox{\plotpoint}}

\put(780.55,557.40){\usebox{\plotpoint}}

\put(797.68,569.12){\usebox{\plotpoint}}

\put(814.24,581.61){\usebox{\plotpoint}}

\put(831.06,593.76){\usebox{\plotpoint}}

\put(847.95,605.82){\usebox{\plotpoint}}

\put(864.61,618.20){\usebox{\plotpoint}}

\put(881.24,630.60){\usebox{\plotpoint}}

\put(898.13,642.67){\usebox{\plotpoint}}

\put(914.99,654.78){\usebox{\plotpoint}}

\put(931.42,667.45){\usebox{\plotpoint}}

\put(948.31,679.51){\usebox{\plotpoint}}

\put(965.20,691.57){\usebox{\plotpoint}}

\put(981.69,704.18){\usebox{\plotpoint}}

\put(998.49,716.35){\usebox{\plotpoint}}

\put(1015.35,728.46){\usebox{\plotpoint}}

\put(1032.14,740.67){\usebox{\plotpoint}}

\put(1049.03,752.73){\usebox{\plotpoint}}

\put(1065.92,764.80){\usebox{\plotpoint}}

\put(1082.34,777.48){\usebox{\plotpoint}}

\put(1099.21,789.58){\usebox{\plotpoint}}

\put(1116.10,801.64){\usebox{\plotpoint}}

\put(1132.72,814.06){\usebox{\plotpoint}}

\put(1149.39,826.42){\usebox{\plotpoint}}

\put(1166.28,838.48){\usebox{\plotpoint}}

\put(1183.09,850.64){\usebox{\plotpoint}}

\put(1199.57,863.26){\usebox{\plotpoint}}

\put(1216.46,875.33){\usebox{\plotpoint}}

\put(1233.00,887.86){\usebox{\plotpoint}}

\put(1249.75,900.11){\usebox{\plotpoint}}

\put(1266.97,911.69){\usebox{\plotpoint}}

\put(1283.73,923.93){\usebox{\plotpoint}}

\put(1300.26,936.47){\usebox{\plotpoint}}

\put(1317.15,948.53){\usebox{\plotpoint}}

\put(1334.04,960.60){\usebox{\plotpoint}}

\put(1350.44,973.31){\usebox{\plotpoint}}

\put(1367.33,985.38){\usebox{\plotpoint}}

\put(1384.01,997.72){\usebox{\plotpoint}}

\put(1400.62,1010.16){\usebox{\plotpoint}}

\put(1417.51,1022.22){\usebox{\plotpoint}}

\put(1434.38,1034.30){\usebox{\plotpoint}}

\put(1450.80,1047.00){\usebox{\plotpoint}}

\put(1467.69,1059.06){\usebox{\plotpoint}}

\put(1484.77,1070.85){\usebox{\plotpoint}}

\put(1501.43,1083.20){\usebox{\plotpoint}}

\put(1518.20,1095.43){\usebox{\plotpoint}}

\put(1535.09,1107.49){\usebox{\plotpoint}}

\put(1551.81,1119.78){\usebox{\plotpoint}}

\put(1568.38,1132.27){\usebox{\plotpoint}}

\put(1585.27,1144.33){\usebox{\plotpoint}}

\put(1601.71,1156.99){\usebox{\plotpoint}}

\put(1603,1158){\usebox{\plotpoint}}

\put(211.0,131.0){\rule[-0.200pt]{0.400pt}{357.014pt}}

\put(211.0,131.0){\rule[-0.200pt]{335.333pt}{0.400pt}}

\put(1603.0,131.0){\rule[-0.200pt]{0.400pt}{357.014pt}}

\put(211.0,1613.0){\rule[-0.200pt]{335.333pt}{0.400pt}}

\end{picture}
    \end{center}
\caption{Beesési ($\alpha$) és törési ($\beta$) szögek kilépéskor}  \end{figure}

Az illesztett egyenes meredeksége $n_{1,2}=\ensuremath{0,686}$. A kapott két törésmutató jó közelítéssel egymás reciproka, $n_{2,1}\cdot n_{1,2} = \ensuremath{1,48} \cdot  \ensuremath{0,69} = \ensuremath{1,02}$.



\section{Törésmutató meghatározása teljes visszaverődés esetén}

Egy prizmán áthaladó fénysugarat vizsgálunk, hogy milyen $\alpha$ beesési szögnél van a teljes visszaverődés határa, vagyik mikor lesz a kilépési szög $90^\circ$. Megmutatható, hogy $\alpha$ alapján a törésmutató a \[ n = \sqrt{\frac{1+2 \cos \phi \sin \alpha + \sin^2 \alpha}{\sin^2 \phi}} \] képlettel meghatározható, ahol $\phi$ a prizma szöge.

A mérés alapján $\alpha = 5^\circ$, ebből $n=\ensuremath{1,504}$.

A teljes visszaverődéshez tartozó határszög

\[n \sin \gamma_h = 1\]
képletből \[\gamma_h = \sin^{-1} \frac 1n = \ensuremath{0,665} = \ensuremath{38,1} ^\circ\]

\end{document}

