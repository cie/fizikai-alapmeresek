\documentclass[12pt]{article}
\usepackage[utf8]{inputenc}
\usepackage[final]{graphics}
\usepackage[pdftex]{graphicx}
\usepackage[dvips]{color}
\usepackage{amsfonts}
\usepackage{subfigure}
\usepackage{lscape}
\usepackage{hyperref}
\usepackage{amsmath}
\usepackage{units}
\usepackage{float}
\usepackage[table]{xcolor}
\usepackage{rmpage}
\usepackage[magyar]{babel}
\usepackage[margin=2cm]{geometry}
\newcommand{\hide}[1]{}
\DeclareGraphicsExtensions{.jpg,.pdf,.mps,.png}
    \title{Optikai alapmérések}
    \author{Kalló Bernát -- Mérés: 2012.05.09. -- Leadás: 2012.06.08.}
    \date{}
    \begin{document}
\def\t1{\ensuremath{t_1}}\def\t2{\ensuremath{t_2}}\def\k1{\ensuremath{k_1}}\def\k2{\ensuremath{k_2}}\def\t11{\ensuremath{\nicefrac{1}{t_1}}}\def\t21{\ensuremath{\nicefrac{1}{t_2}}}\def\k11{\ensuremath{\nicefrac{1}{k_1}}}\def\k21{\ensuremath{\nicefrac{1}{k_2}}}\def\tx1{\ensuremath{\nicefrac{1}{t}}}\def\kx1{\ensuremath{\nicefrac{1}{k}}}\def\dmin1{\ensuremath{d_{min,1}}}\def\dmin2{\ensuremath{d_{min,2}}}\def\y1{\ensuremath{y_{1}}}\def\y2{\ensuremath{y_{2}}}\def\a1{\ensuremath{a_{1}}}\def\a2{\ensuremath{a_{2}}}\def\deltaa1{\ensuremath{\delta{}a_{1}}}\def\deltaa2{\ensuremath{\delta{}a_{2}}}

\maketitle

\section{Törésmutató meghatározása}

Két közeg határán áthaladó fénysugár megtörik, úgy, hogy a beesési és törési szögek szinuszának aránya \[\frac{\sin\alpha}{\sin\beta} = n_{2,1}\] állandó, a két közeg relatív törésmutatója.

Megmértük egy félkör alakú lencse egyenes oldalán belépő fénysugár törési szögét. Valójában a kilépő fénysugár szögét mértük, de mivel a görbe oldalon való kilépéskor a sugár merőleges a határfelületre, az iránya ott nem változik meg. A mért adatok az alábbi táblázatban láthatók:

  
  \begin{table}[H]
    \begin{center}
      \begin{tabular}{|
c|
c|
}
        \hline
        
\ensuremath{\unit[\text{\ensuremath{\alpha}}]{({}^\circ)}} & 
\ensuremath{\unit[\text{\ensuremath{\beta}}]{({}^\circ)}}
\\
        \hline\hline
        
0
 & 0
\\
        \hline
        
10
 & 7
\\
        \hline
        
20
 & 13
\\
        \hline
        
30
 & 20
\\
        \hline
        
40
 & 26
\\
        \hline
        
50
 & 31
\\
        \hline
        
60
 & 36
\\
        \hline
        
70
 & 39
\\
        \hline
        
80
 & 42
\\
        \hline
      \end{tabular}
      \caption{Beesési és törési szögek belépéskor}
      \label{tab:}
    \end{center}
  \end{table}

Ezt grafikonon ábrázolva:

  \begin{figure}[H]
    \begin{center}
% GNUPLOT: LaTeX picture

\setlength{\unitlength}{0.240900pt}

\ifx\plotpoint\undefined\newsavebox{\plotpoint}\fi

\sbox{\plotpoint}{\rule[-0.200pt]{0.400pt}{0.400pt}}%

\begin{picture}(1653,1653)(0,0)

\sbox{\plotpoint}{\rule[-0.200pt]{0.400pt}{0.400pt}}%

\put(211.0,131.0){\rule[-0.200pt]{4.818pt}{0.400pt}}

\put(191,131){\makebox(0,0)[r]{ 0}}

\put(1583.0,131.0){\rule[-0.200pt]{4.818pt}{0.400pt}}

\put(211.0,427.0){\rule[-0.200pt]{4.818pt}{0.400pt}}

\put(191,427){\makebox(0,0)[r]{ 0.2}}

\put(1583.0,427.0){\rule[-0.200pt]{4.818pt}{0.400pt}}

\put(211.0,724.0){\rule[-0.200pt]{4.818pt}{0.400pt}}

\put(191,724){\makebox(0,0)[r]{ 0.4}}

\put(1583.0,724.0){\rule[-0.200pt]{4.818pt}{0.400pt}}

\put(211.0,1020.0){\rule[-0.200pt]{4.818pt}{0.400pt}}

\put(191,1020){\makebox(0,0)[r]{ 0.6}}

\put(1583.0,1020.0){\rule[-0.200pt]{4.818pt}{0.400pt}}

\put(211.0,1317.0){\rule[-0.200pt]{4.818pt}{0.400pt}}

\put(191,1317){\makebox(0,0)[r]{ 0.8}}

\put(1583.0,1317.0){\rule[-0.200pt]{4.818pt}{0.400pt}}

\put(211.0,1613.0){\rule[-0.200pt]{4.818pt}{0.400pt}}

\put(191,1613){\makebox(0,0)[r]{ 1}}

\put(1583.0,1613.0){\rule[-0.200pt]{4.818pt}{0.400pt}}

\put(211.0,131.0){\rule[-0.200pt]{0.400pt}{4.818pt}}

\put(211,90){\makebox(0,0){ 0}}

\put(211.0,1593.0){\rule[-0.200pt]{0.400pt}{4.818pt}}

\put(489.0,131.0){\rule[-0.200pt]{0.400pt}{4.818pt}}

\put(489,90){\makebox(0,0){ 0.2}}

\put(489.0,1593.0){\rule[-0.200pt]{0.400pt}{4.818pt}}

\put(768.0,131.0){\rule[-0.200pt]{0.400pt}{4.818pt}}

\put(768,90){\makebox(0,0){ 0.4}}

\put(768.0,1593.0){\rule[-0.200pt]{0.400pt}{4.818pt}}

\put(1046.0,131.0){\rule[-0.200pt]{0.400pt}{4.818pt}}

\put(1046,90){\makebox(0,0){ 0.6}}

\put(1046.0,1593.0){\rule[-0.200pt]{0.400pt}{4.818pt}}

\put(1325.0,131.0){\rule[-0.200pt]{0.400pt}{4.818pt}}

\put(1325,90){\makebox(0,0){ 0.8}}

\put(1325.0,1593.0){\rule[-0.200pt]{0.400pt}{4.818pt}}

\put(1603.0,131.0){\rule[-0.200pt]{0.400pt}{4.818pt}}

\put(1603,90){\makebox(0,0){ 1}}

\put(1603.0,1593.0){\rule[-0.200pt]{0.400pt}{4.818pt}}

\put(211.0,131.0){\rule[-0.200pt]{0.400pt}{357.014pt}}

\put(211.0,131.0){\rule[-0.200pt]{335.333pt}{0.400pt}}

\put(1603.0,131.0){\rule[-0.200pt]{0.400pt}{357.014pt}}

\put(211.0,1613.0){\rule[-0.200pt]{335.333pt}{0.400pt}}

\put(70,872){\makebox(0,0){$\sin\alpha$}}

\put(907,29){\makebox(0,0){$\sin\beta$}}

\put(211,131){\raisebox{-.8pt}{\makebox(0,0){$\Diamond$}}}

\put(381,388){\raisebox{-.8pt}{\makebox(0,0){$\Diamond$}}}

\put(524,638){\raisebox{-.8pt}{\makebox(0,0){$\Diamond$}}}

\put(687,872){\raisebox{-.8pt}{\makebox(0,0){$\Diamond$}}}

\put(821,1084){\raisebox{-.8pt}{\makebox(0,0){$\Diamond$}}}

\put(928,1266){\raisebox{-.8pt}{\makebox(0,0){$\Diamond$}}}

\put(1029,1414){\raisebox{-.8pt}{\makebox(0,0){$\Diamond$}}}

\put(1087,1524){\raisebox{-.8pt}{\makebox(0,0){$\Diamond$}}}

\put(1142,1590){\raisebox{-.8pt}{\makebox(0,0){$\Diamond$}}}

\multiput(212,131)(11.312,17.402){2}{\usebox{\plotpoint}}

\put(234.18,166.08){\usebox{\plotpoint}}

\put(245.17,183.69){\usebox{\plotpoint}}

\put(256.31,201.20){\usebox{\plotpoint}}

\multiput(267,218)(11.143,17.511){2}{\usebox{\plotpoint}}

\put(289.74,253.73){\usebox{\plotpoint}}

\put(300.88,271.24){\usebox{\plotpoint}}

\multiput(309,284)(10.792,17.729){2}{\usebox{\plotpoint}}

\put(334.39,323.71){\usebox{\plotpoint}}

\put(345.70,341.11){\usebox{\plotpoint}}

\put(356.85,358.62){\usebox{\plotpoint}}

\multiput(366,373)(11.143,17.511){2}{\usebox{\plotpoint}}

\put(390.28,411.15){\usebox{\plotpoint}}

\put(401.19,428.80){\usebox{\plotpoint}}

\put(412.11,446.45){\usebox{\plotpoint}}

\multiput(422,462)(11.143,17.511){2}{\usebox{\plotpoint}}

\put(445.54,498.99){\usebox{\plotpoint}}

\put(456.68,516.50){\usebox{\plotpoint}}

\put(467.82,534.01){\usebox{\plotpoint}}

\multiput(478,550)(10.792,17.729){2}{\usebox{\plotpoint}}

\put(500.80,586.82){\usebox{\plotpoint}}

\put(511.94,604.33){\usebox{\plotpoint}}

\put(523.08,621.84){\usebox{\plotpoint}}

\multiput(534,639)(11.143,17.511){2}{\usebox{\plotpoint}}

\put(556.93,674.10){\usebox{\plotpoint}}

\put(568.19,691.53){\usebox{\plotpoint}}

\multiput(577,706)(11.143,17.511){2}{\usebox{\plotpoint}}

\put(601.33,744.24){\usebox{\plotpoint}}

\put(612.48,761.75){\usebox{\plotpoint}}

\put(623.62,779.26){\usebox{\plotpoint}}

\multiput(633,794)(10.792,17.729){2}{\usebox{\plotpoint}}

\put(656.59,832.07){\usebox{\plotpoint}}

\put(667.74,849.58){\usebox{\plotpoint}}

\put(678.88,867.10){\usebox{\plotpoint}}

\multiput(689,883)(11.143,17.511){2}{\usebox{\plotpoint}}

\put(712.31,919.63){\usebox{\plotpoint}}

\put(723.25,937.26){\usebox{\plotpoint}}

\put(734.14,954.93){\usebox{\plotpoint}}

\multiput(745,972)(11.143,17.511){2}{\usebox{\plotpoint}}

\put(767.57,1007.46){\usebox{\plotpoint}}

\put(778.71,1024.97){\usebox{\plotpoint}}

\multiput(787,1038)(11.692,17.149){2}{\usebox{\plotpoint}}

\put(812.50,1077.26){\usebox{\plotpoint}}

\put(823.53,1094.84){\usebox{\plotpoint}}

\put(834.68,1112.35){\usebox{\plotpoint}}

\multiput(844,1127)(11.143,17.511){2}{\usebox{\plotpoint}}

\put(868.11,1164.88){\usebox{\plotpoint}}

\put(879.25,1182.39){\usebox{\plotpoint}}

\put(890.25,1199.99){\usebox{\plotpoint}}

\multiput(900,1216)(11.143,17.511){2}{\usebox{\plotpoint}}

\put(923.36,1252.72){\usebox{\plotpoint}}

\put(934.51,1270.23){\usebox{\plotpoint}}

\put(945.65,1287.74){\usebox{\plotpoint}}

\multiput(956,1304)(11.143,17.511){2}{\usebox{\plotpoint}}

\put(978.79,1340.45){\usebox{\plotpoint}}

\put(989.77,1358.06){\usebox{\plotpoint}}

\put(1000.91,1375.57){\usebox{\plotpoint}}

\multiput(1012,1393)(11.692,17.149){2}{\usebox{\plotpoint}}

\put(1035.04,1427.64){\usebox{\plotpoint}}

\put(1046.19,1445.15){\usebox{\plotpoint}}

\multiput(1055,1459)(10.792,17.729){2}{\usebox{\plotpoint}}

\put(1079.16,1497.97){\usebox{\plotpoint}}

\put(1090.30,1515.48){\usebox{\plotpoint}}

\put(1101.45,1532.99){\usebox{\plotpoint}}

\multiput(1111,1548)(11.143,17.511){2}{\usebox{\plotpoint}}

\put(1134.88,1585.52){\usebox{\plotpoint}}

\put(1145.88,1603.12){\usebox{\plotpoint}}

\put(1152,1613){\usebox{\plotpoint}}

\put(211.0,131.0){\rule[-0.200pt]{0.400pt}{357.014pt}}

\put(211.0,131.0){\rule[-0.200pt]{335.333pt}{0.400pt}}

\put(1603.0,131.0){\rule[-0.200pt]{0.400pt}{357.014pt}}

\put(211.0,1613.0){\rule[-0.200pt]{335.333pt}{0.400pt}}

\end{picture}
    \end{center}
\caption{Beesési ($\alpha$) és törési ($\beta$) szögek belépéskor}  \end{figure}

Az illesztett egyenes meredeksége $n_{2,1} = \ensuremath{1,481}$, tehát ez a műanyagnak a levegőre vonatkoztatott törésmutatója.

Ezután elvégeztük a kísérletet fordítva is, a kilépéskor történő törést mérve.
  
  \begin{table}[H]
    \begin{center}
      \begin{tabular}{|
c|
c|
}
        \hline
        
\ensuremath{\unit[\text{\ensuremath{\alpha}}]{({}^\circ)}} & 
\ensuremath{\unit[\text{\ensuremath{\beta}}]{({}^\circ)}}
\\
        \hline\hline
        
0
 & 0
\\
        \hline
        
10
 & 14
\\
        \hline
        
15
 & 21
\\
        \hline
        
20
 & 29
\\
        \hline
        
25
 & 37
\\
        \hline
        
30
 & 46
\\
        \hline
        
35
 & 56
\\
        \hline
        
40
 & 69
\\
        \hline
      \end{tabular}
      \caption{Beesési és törési szögek kilépéskor}
      \label{tab:}
    \end{center}
  \end{table}

Grafikonon ábrázolva:

  \begin{figure}[H]
    \begin{center}
% GNUPLOT: LaTeX picture

\setlength{\unitlength}{0.240900pt}

\ifx\plotpoint\undefined\newsavebox{\plotpoint}\fi

\sbox{\plotpoint}{\rule[-0.200pt]{0.400pt}{0.400pt}}%

\begin{picture}(1653,1653)(0,0)

\sbox{\plotpoint}{\rule[-0.200pt]{0.400pt}{0.400pt}}%

\put(211.0,131.0){\rule[-0.200pt]{4.818pt}{0.400pt}}

\put(191,131){\makebox(0,0)[r]{ 0}}

\put(1583.0,131.0){\rule[-0.200pt]{4.818pt}{0.400pt}}

\put(211.0,427.0){\rule[-0.200pt]{4.818pt}{0.400pt}}

\put(191,427){\makebox(0,0)[r]{ 0.2}}

\put(1583.0,427.0){\rule[-0.200pt]{4.818pt}{0.400pt}}

\put(211.0,724.0){\rule[-0.200pt]{4.818pt}{0.400pt}}

\put(191,724){\makebox(0,0)[r]{ 0.4}}

\put(1583.0,724.0){\rule[-0.200pt]{4.818pt}{0.400pt}}

\put(211.0,1020.0){\rule[-0.200pt]{4.818pt}{0.400pt}}

\put(191,1020){\makebox(0,0)[r]{ 0.6}}

\put(1583.0,1020.0){\rule[-0.200pt]{4.818pt}{0.400pt}}

\put(211.0,1317.0){\rule[-0.200pt]{4.818pt}{0.400pt}}

\put(191,1317){\makebox(0,0)[r]{ 0.8}}

\put(1583.0,1317.0){\rule[-0.200pt]{4.818pt}{0.400pt}}

\put(211.0,1613.0){\rule[-0.200pt]{4.818pt}{0.400pt}}

\put(191,1613){\makebox(0,0)[r]{ 1}}

\put(1583.0,1613.0){\rule[-0.200pt]{4.818pt}{0.400pt}}

\put(211.0,131.0){\rule[-0.200pt]{0.400pt}{4.818pt}}

\put(211,90){\makebox(0,0){ 0}}

\put(211.0,1593.0){\rule[-0.200pt]{0.400pt}{4.818pt}}

\put(489.0,131.0){\rule[-0.200pt]{0.400pt}{4.818pt}}

\put(489,90){\makebox(0,0){ 0.2}}

\put(489.0,1593.0){\rule[-0.200pt]{0.400pt}{4.818pt}}

\put(768.0,131.0){\rule[-0.200pt]{0.400pt}{4.818pt}}

\put(768,90){\makebox(0,0){ 0.4}}

\put(768.0,1593.0){\rule[-0.200pt]{0.400pt}{4.818pt}}

\put(1046.0,131.0){\rule[-0.200pt]{0.400pt}{4.818pt}}

\put(1046,90){\makebox(0,0){ 0.6}}

\put(1046.0,1593.0){\rule[-0.200pt]{0.400pt}{4.818pt}}

\put(1325.0,131.0){\rule[-0.200pt]{0.400pt}{4.818pt}}

\put(1325,90){\makebox(0,0){ 0.8}}

\put(1325.0,1593.0){\rule[-0.200pt]{0.400pt}{4.818pt}}

\put(1603.0,131.0){\rule[-0.200pt]{0.400pt}{4.818pt}}

\put(1603,90){\makebox(0,0){ 1}}

\put(1603.0,1593.0){\rule[-0.200pt]{0.400pt}{4.818pt}}

\put(211.0,131.0){\rule[-0.200pt]{0.400pt}{357.014pt}}

\put(211.0,131.0){\rule[-0.200pt]{335.333pt}{0.400pt}}

\put(1603.0,131.0){\rule[-0.200pt]{0.400pt}{357.014pt}}

\put(211.0,1613.0){\rule[-0.200pt]{335.333pt}{0.400pt}}

\put(70,872){\makebox(0,0){$\sin\alpha$}}

\put(907,29){\makebox(0,0){$\sin\beta$}}

\put(211,131){\raisebox{-.8pt}{\makebox(0,0){$\Diamond$}}}

\put(548,388){\raisebox{-.8pt}{\makebox(0,0){$\Diamond$}}}

\put(710,515){\raisebox{-.8pt}{\makebox(0,0){$\Diamond$}}}

\put(886,638){\raisebox{-.8pt}{\makebox(0,0){$\Diamond$}}}

\put(1049,757){\raisebox{-.8pt}{\makebox(0,0){$\Diamond$}}}

\put(1212,872){\raisebox{-.8pt}{\makebox(0,0){$\Diamond$}}}

\put(1365,981){\raisebox{-.8pt}{\makebox(0,0){$\Diamond$}}}

\put(1511,1084){\raisebox{-.8pt}{\makebox(0,0){$\Diamond$}}}

\put(211,141){\usebox{\plotpoint}}

\put(211.00,141.00){\usebox{\plotpoint}}

\put(227.79,153.19){\usebox{\plotpoint}}

\put(244.29,165.78){\usebox{\plotpoint}}

\put(261.18,177.84){\usebox{\plotpoint}}

\put(278.07,189.91){\usebox{\plotpoint}}

\put(294.49,202.60){\usebox{\plotpoint}}

\put(311.36,214.69){\usebox{\plotpoint}}

\put(328.37,226.58){\usebox{\plotpoint}}

\put(345.22,238.67){\usebox{\plotpoint}}

\put(361.87,251.05){\usebox{\plotpoint}}

\put(378.76,263.11){\usebox{\plotpoint}}

\put(395.59,275.25){\usebox{\plotpoint}}

\put(412.05,287.89){\usebox{\plotpoint}}

\put(428.94,299.96){\usebox{\plotpoint}}

\put(445.50,312.46){\usebox{\plotpoint}}

\put(462.23,324.74){\usebox{\plotpoint}}

\put(479.12,336.80){\usebox{\plotpoint}}

\put(495.88,349.04){\usebox{\plotpoint}}

\put(512.41,361.58){\usebox{\plotpoint}}

\put(529.30,373.64){\usebox{\plotpoint}}

\put(546.19,385.71){\usebox{\plotpoint}}

\put(562.94,397.96){\usebox{\plotpoint}}

\put(579.83,410.02){\usebox{\plotpoint}}

\put(596.72,422.09){\usebox{\plotpoint}}

\put(613.32,434.54){\usebox{\plotpoint}}

\put(630.01,446.87){\usebox{\plotpoint}}

\put(646.90,458.93){\usebox{\plotpoint}}

\put(663.30,471.65){\usebox{\plotpoint}}

\put(680.19,483.71){\usebox{\plotpoint}}

\put(697.08,495.77){\usebox{\plotpoint}}

\put(713.60,508.33){\usebox{\plotpoint}}

\put(730.37,520.55){\usebox{\plotpoint}}

\put(747.26,532.62){\usebox{\plotpoint}}

\put(763.98,544.91){\usebox{\plotpoint}}

\put(780.55,557.40){\usebox{\plotpoint}}

\put(797.68,569.12){\usebox{\plotpoint}}

\put(814.24,581.61){\usebox{\plotpoint}}

\put(831.06,593.76){\usebox{\plotpoint}}

\put(847.95,605.82){\usebox{\plotpoint}}

\put(864.61,618.20){\usebox{\plotpoint}}

\put(881.24,630.60){\usebox{\plotpoint}}

\put(898.13,642.67){\usebox{\plotpoint}}

\put(914.99,654.78){\usebox{\plotpoint}}

\put(931.42,667.45){\usebox{\plotpoint}}

\put(948.31,679.51){\usebox{\plotpoint}}

\put(965.20,691.57){\usebox{\plotpoint}}

\put(981.69,704.18){\usebox{\plotpoint}}

\put(998.49,716.35){\usebox{\plotpoint}}

\put(1015.35,728.46){\usebox{\plotpoint}}

\put(1032.14,740.67){\usebox{\plotpoint}}

\put(1049.03,752.73){\usebox{\plotpoint}}

\put(1065.92,764.80){\usebox{\plotpoint}}

\put(1082.34,777.48){\usebox{\plotpoint}}

\put(1099.21,789.58){\usebox{\plotpoint}}

\put(1116.10,801.64){\usebox{\plotpoint}}

\put(1132.72,814.06){\usebox{\plotpoint}}

\put(1149.39,826.42){\usebox{\plotpoint}}

\put(1166.28,838.48){\usebox{\plotpoint}}

\put(1183.09,850.64){\usebox{\plotpoint}}

\put(1199.57,863.26){\usebox{\plotpoint}}

\put(1216.46,875.33){\usebox{\plotpoint}}

\put(1233.00,887.86){\usebox{\plotpoint}}

\put(1249.75,900.11){\usebox{\plotpoint}}

\put(1266.97,911.69){\usebox{\plotpoint}}

\put(1283.73,923.93){\usebox{\plotpoint}}

\put(1300.26,936.47){\usebox{\plotpoint}}

\put(1317.15,948.53){\usebox{\plotpoint}}

\put(1334.04,960.60){\usebox{\plotpoint}}

\put(1350.44,973.31){\usebox{\plotpoint}}

\put(1367.33,985.38){\usebox{\plotpoint}}

\put(1384.01,997.72){\usebox{\plotpoint}}

\put(1400.62,1010.16){\usebox{\plotpoint}}

\put(1417.51,1022.22){\usebox{\plotpoint}}

\put(1434.38,1034.30){\usebox{\plotpoint}}

\put(1450.80,1047.00){\usebox{\plotpoint}}

\put(1467.69,1059.06){\usebox{\plotpoint}}

\put(1484.77,1070.85){\usebox{\plotpoint}}

\put(1501.43,1083.20){\usebox{\plotpoint}}

\put(1518.20,1095.43){\usebox{\plotpoint}}

\put(1535.09,1107.49){\usebox{\plotpoint}}

\put(1551.81,1119.78){\usebox{\plotpoint}}

\put(1568.38,1132.27){\usebox{\plotpoint}}

\put(1585.27,1144.33){\usebox{\plotpoint}}

\put(1601.71,1156.99){\usebox{\plotpoint}}

\put(1603,1158){\usebox{\plotpoint}}

\put(211.0,131.0){\rule[-0.200pt]{0.400pt}{357.014pt}}

\put(211.0,131.0){\rule[-0.200pt]{335.333pt}{0.400pt}}

\put(1603.0,131.0){\rule[-0.200pt]{0.400pt}{357.014pt}}

\put(211.0,1613.0){\rule[-0.200pt]{335.333pt}{0.400pt}}

\end{picture}
    \end{center}
\caption{Beesési ($\alpha$) és törési ($\beta$) szögek kilépéskor}  \end{figure}

Az illesztett egyenes meredeksége $n_{1,2}=\ensuremath{0,686}$. A kapott két törésmutató jó közelítéssel egymás reciproka, $n_{2,1}\cdot n_{1,2} = \ensuremath{1,48} \cdot  \ensuremath{0,69} = \ensuremath{1,02}$.



\section{Törésmutató meghatározása teljes visszaverődés esetén}

Egy prizmán áthaladó fénysugarat vizsgálunk, hogy milyen $\alpha$ beesési szögnél van a teljes visszaverődés határa, vagyik mikor lesz a kilépési szög $90^\circ$. Megmutatható, hogy $\alpha$ alapján a törésmutató a \[ n = \sqrt{\frac{1+2 \cos \phi \sin \alpha + \sin^2 \alpha}{\sin^2 \phi}} \] képlettel meghatározható, ahol $\phi$ a prizma szöge.

A mérés alapján $\alpha = 5^\circ$, ebből $n=\ensuremath{1,504}$.

A teljes visszaverődéshez tartozó határszög

\[n \sin \gamma_h = 1\]
képletből \[\gamma_h = \sin^{-1} \frac 1n = \ensuremath{0,665} = \ensuremath{38,1} ^\circ.\]

Ahogy a beesési szöget csökkentettük, a piros fény lépett ki legkorábban, tehát ennek volt a legnagyobb a határszöge. Vagyis a piros fényre a legkisebb a törésmutató, ezek szerint a lila fényre a legnagyobb, a lila fény térül el a legjobban.


\section{Lencse törőképességének vizsgálata}

Egy három tartályból álló átlátszó műanyag edényt, a tartályokat egyenként megtöltve vízzel, lencseként használunk.


\vskip 50mm

Párhuzamos fénysugarakat engedünk át a lencsén, és az alábbiakat tapasztaljuk.

  
  \begin{table}[H]
    \begin{center}
      \begin{tabular}{|
c|
c|
c|
c|
c|
}
        \hline
        
 & 
1 & 2 & 3 & a fénysugarak
\\
        \hline\hline
        
1
 & levegő
 & levegő
 & levegő
 & párhuzamosak
\\
        \hline
        
a
 & víz
 & levegő
 & levegő
 & széttartanak
\\
        \hline
        
b
 & levegő
 & víz
 & levegő
 & összetartanak
\\
        \hline
        
c
 & levegő
 & levegő
 & víz
 & összetartanak
\\
        \hline
        
d
 & víz
 & levegő
 & víz
 & enyhén széttartanak
\\
        \hline
      \end{tabular}
      \caption{A fénysugarak}
      \label{tab:}
    \end{center}
  \end{table}

Amikor nincs víz a tartályokban, a fény párhuzamosan halad át. Mivel mindhárom tartályban levegő van, a műanyagba belépéskor és kilépéskor azonos a törésmutató. A műanyag falak vékonyak, így a belépési és a kilépési felület nagyjából párhuzamos, tehát befelé haladva törőszög megegyezik kifelé haladva a beesési szöggel, és a törésmutatók egyenlősége miatt a beesési és a kilépési szögek is egyenlőek. 

Amikor valamelyik tartályban víz van, a műanyag két oldalán nem egyenlő a törésmutató. Viszont mivel a műanyag vékony és a falai párhuzamosak, a fénysugárnak a merőlegeshez viszonyított szögének szinusza először a műanyag és levegő, másodszor a víz és a műanyag relatív törésmutatójával változik. De e két törésmutató szorzata egyenlő a víz és a levegő törésmutatójával. Tehát tekinthetjük úgy, mintha a műanyag nem is lenne ott, csak a víz lenne, mint egy lencse.

Az a) esetben a víz egy sima-homoró lencsét alkot, ez szórólencse. A b) esetben domború-domború lencsét, ez gyűjtőlencse. A c) esetben domború-sima lecsét, ez is gyűjtőlencse. A d) esetben két lencsét alkot, egy gyűjtő és egy szórólencsét. Viszont a szórólencse görbülete nagyobb, tehát erősebb, mint a gyűjtőlencse, ezért a két lencséből álló rendszer szétszórja a sugarakat, bár kevébé, mint a szórólencse egyedül az a) esetben.


\section{Gyűjtőlencse fókusztávolsága}

Egy fényforrás és egy ernyő közé gyűjtőlencsét helyezünk. Megkeressük azt a két pozícióját a lencsének, amiben az ernyőre vetülő kép éles. Amikor a fényforráshoz van közelebb a lencse, a kép fordított állású és nagyított. Amikor az ernyőhöz van közel a lencse, a kép fordított állású és kicsinyített.

Különböző fényforrás-ernyő távolságokkal ($l$) is megmérjük a két lehetséges tárgy-és képtávolságot:

  
  \begin{table}[H]
    \begin{center}
      \begin{tabular}{|
c|
c|
c|
c|
c|
}
        \hline
        
\ensuremath{\unit[\text{\ensuremath{l}}]{(m)}} & 
\ensuremath{\unit[\text{\ensuremath{t_1}}]{(m)}} & \ensuremath{\unit[\text{\ensuremath{k_1}}]{(m)}} & \ensuremath{\unit[\text{\ensuremath{t_2}}]{(m)}} & \ensuremath{\unit[\text{\ensuremath{k_2}}]{(m)}}
\\
        \hline\hline
        
1.0
 & 0,122
 & 0,878
 & 0,880
 & 0,120
\\
        \hline
        
0.9
 & 0,124
 & 0,776
 & 0,780
 & 0,120
\\
        \hline
        
0.8
 & 0,125
 & 0,675
 & 0,674
 & 0,126
\\
        \hline
        
0.7
 & 0,131
 & 0,569
 & 0,571
 & 0,129
\\
        \hline
        
0.6
 & 0,137
 & 0,463
 & 0,464
 & 0,136
\\
        \hline
        
0.5
 & 0,148
 & 0,352
 & 0,349
 & 0,151
\\
        \hline
      \end{tabular}
      \caption{A mért adatok}
      \label{tab:}
    \end{center}
  \end{table}

Kiszámítjuk a reciprok értékeket:

  
  \begin{table}[H]
    \begin{center}
      \begin{tabular}{|
c|
c|
c|
c|
c|
}
        \hline
        
\ensuremath{\unit[\text{\ensuremath{l}}]{(m)}} & 
\ensuremath{\unit[\text{\ensuremath{\nicefrac{1}{t_1}}}]{(m^{-1})}} & \ensuremath{\unit[\text{\ensuremath{\nicefrac{1}{k_1}}}]{(m^{-1})}} & \ensuremath{\unit[\text{\ensuremath{\nicefrac{1}{t_2}}}]{(m^{-1})}} & \ensuremath{\unit[\text{\ensuremath{\nicefrac{1}{k_2}}}]{(m^{-1})}}
\\
        \hline\hline
        
1.0
 & 8,20
 & 1,139
 & 1,136
 & 8,33
\\
        \hline
        
0.9
 & 8,06
 & 1,289
 & 1,282
 & 8,33
\\
        \hline
        
0.8
 & 8,00
 & 1,481
 & 1,484
 & 7,94
\\
        \hline
        
0.7
 & 7,63
 & 1,757
 & 1,751
 & 7,75
\\
        \hline
        
0.6
 & 7,30
 & 2,160
 & 2,155
 & 7,35
\\
        \hline
        
0.5
 & 6,76
 & 2,841
 & 2,865
 & 6,62
\\
        \hline
      \end{tabular}
      \caption{A mért adatok}
      \label{tab:}
    \end{center}
  \end{table}

És ábrázoljuk őket grafikonon:

  \begin{figure}[H]
    \begin{center}
% GNUPLOT: LaTeX picture

\setlength{\unitlength}{0.240900pt}

\ifx\plotpoint\undefined\newsavebox{\plotpoint}\fi

\sbox{\plotpoint}{\rule[-0.200pt]{0.400pt}{0.400pt}}%

\begin{picture}(1771,1771)(0,0)

\sbox{\plotpoint}{\rule[-0.200pt]{0.400pt}{0.400pt}}%

\put(191.0,131.0){\rule[-0.200pt]{4.818pt}{0.400pt}}

\put(171,131){\makebox(0,0)[r]{ 0}}

\put(1701.0,131.0){\rule[-0.200pt]{4.818pt}{0.400pt}}

\put(191.0,451.0){\rule[-0.200pt]{4.818pt}{0.400pt}}

\put(171,451){\makebox(0,0)[r]{ 2}}

\put(1701.0,451.0){\rule[-0.200pt]{4.818pt}{0.400pt}}

\put(191.0,771.0){\rule[-0.200pt]{4.818pt}{0.400pt}}

\put(171,771){\makebox(0,0)[r]{ 4}}

\put(1701.0,771.0){\rule[-0.200pt]{4.818pt}{0.400pt}}

\put(191.0,1091.0){\rule[-0.200pt]{4.818pt}{0.400pt}}

\put(171,1091){\makebox(0,0)[r]{ 6}}

\put(1701.0,1091.0){\rule[-0.200pt]{4.818pt}{0.400pt}}

\put(191.0,1411.0){\rule[-0.200pt]{4.818pt}{0.400pt}}

\put(171,1411){\makebox(0,0)[r]{ 8}}

\put(1701.0,1411.0){\rule[-0.200pt]{4.818pt}{0.400pt}}

\put(191.0,1731.0){\rule[-0.200pt]{4.818pt}{0.400pt}}

\put(171,1731){\makebox(0,0)[r]{ 10}}

\put(1701.0,1731.0){\rule[-0.200pt]{4.818pt}{0.400pt}}

\put(191.0,131.0){\rule[-0.200pt]{0.400pt}{4.818pt}}

\put(191,90){\makebox(0,0){ 0}}

\put(191.0,1711.0){\rule[-0.200pt]{0.400pt}{4.818pt}}

\put(497.0,131.0){\rule[-0.200pt]{0.400pt}{4.818pt}}

\put(497,90){\makebox(0,0){ 2}}

\put(497.0,1711.0){\rule[-0.200pt]{0.400pt}{4.818pt}}

\put(803.0,131.0){\rule[-0.200pt]{0.400pt}{4.818pt}}

\put(803,90){\makebox(0,0){ 4}}

\put(803.0,1711.0){\rule[-0.200pt]{0.400pt}{4.818pt}}

\put(1109.0,131.0){\rule[-0.200pt]{0.400pt}{4.818pt}}

\put(1109,90){\makebox(0,0){ 6}}

\put(1109.0,1711.0){\rule[-0.200pt]{0.400pt}{4.818pt}}

\put(1415.0,131.0){\rule[-0.200pt]{0.400pt}{4.818pt}}

\put(1415,90){\makebox(0,0){ 8}}

\put(1415.0,1711.0){\rule[-0.200pt]{0.400pt}{4.818pt}}

\put(1721.0,131.0){\rule[-0.200pt]{0.400pt}{4.818pt}}

\put(1721,90){\makebox(0,0){ 10}}

\put(1721.0,1711.0){\rule[-0.200pt]{0.400pt}{4.818pt}}

\put(191.0,131.0){\rule[-0.200pt]{0.400pt}{385.440pt}}

\put(191.0,131.0){\rule[-0.200pt]{368.577pt}{0.400pt}}

\put(1721.0,131.0){\rule[-0.200pt]{0.400pt}{385.440pt}}

\put(191.0,1731.0){\rule[-0.200pt]{368.577pt}{0.400pt}}

\put(70,931){\makebox(0,0){\ensuremath{\unit[\text{\ensuremath{\nicefrac{1}{k}}}]{(m^{-1})}}}}

\put(956,29){\makebox(0,0){\ensuremath{\unit[\text{\ensuremath{\nicefrac{1}{t}}}]{(m^{-1})}}}}

\put(1561,1691){\makebox(0,0)[r]{1.}}

\put(1445,313){\raisebox{-.8pt}{\makebox(0,0){$\Diamond$}}}

\put(1425,337){\raisebox{-.8pt}{\makebox(0,0){$\Diamond$}}}

\put(1415,368){\raisebox{-.8pt}{\makebox(0,0){$\Diamond$}}}

\put(1359,412){\raisebox{-.8pt}{\makebox(0,0){$\Diamond$}}}

\put(1308,477){\raisebox{-.8pt}{\makebox(0,0){$\Diamond$}}}

\put(1225,586){\raisebox{-.8pt}{\makebox(0,0){$\Diamond$}}}

\put(1631,1691){\raisebox{-.8pt}{\makebox(0,0){$\Diamond$}}}

\put(1561,1650){\makebox(0,0)[r]{2.}}

\put(365,1464){\makebox(0,0){$+$}}

\put(387,1464){\makebox(0,0){$+$}}

\put(418,1401){\makebox(0,0){$+$}}

\put(459,1371){\makebox(0,0){$+$}}

\put(521,1307){\makebox(0,0){$+$}}

\put(629,1191){\makebox(0,0){$+$}}

\put(1631,1650){\makebox(0,0){$+$}}

\sbox{\plotpoint}{\rule[-0.400pt]{0.800pt}{0.800pt}}%

\put(191,1656){\usebox{\plotpoint}}

\multiput(192.41,1651.41)(0.508,-0.564){23}{\rule{0.122pt}{1.107pt}}

\multiput(189.34,1653.70)(15.000,-14.703){2}{\rule{0.800pt}{0.553pt}}

\multiput(206.00,1637.09)(0.494,-0.507){25}{\rule{1.000pt}{0.122pt}}

\multiput(206.00,1637.34)(13.924,-16.000){2}{\rule{0.500pt}{0.800pt}}

\multiput(223.41,1618.63)(0.508,-0.529){23}{\rule{0.122pt}{1.053pt}}

\multiput(220.34,1620.81)(15.000,-13.814){2}{\rule{0.800pt}{0.527pt}}

\multiput(238.41,1602.64)(0.507,-0.527){25}{\rule{0.122pt}{1.050pt}}

\multiput(235.34,1604.82)(16.000,-14.821){2}{\rule{0.800pt}{0.525pt}}

\multiput(254.41,1585.63)(0.508,-0.529){23}{\rule{0.122pt}{1.053pt}}

\multiput(251.34,1587.81)(15.000,-13.814){2}{\rule{0.800pt}{0.527pt}}

\multiput(269.41,1569.64)(0.507,-0.527){25}{\rule{0.122pt}{1.050pt}}

\multiput(266.34,1571.82)(16.000,-14.821){2}{\rule{0.800pt}{0.525pt}}

\multiput(285.41,1552.63)(0.508,-0.529){23}{\rule{0.122pt}{1.053pt}}

\multiput(282.34,1554.81)(15.000,-13.814){2}{\rule{0.800pt}{0.527pt}}

\multiput(299.00,1539.09)(0.494,-0.507){25}{\rule{1.000pt}{0.122pt}}

\multiput(299.00,1539.34)(13.924,-16.000){2}{\rule{0.500pt}{0.800pt}}

\multiput(316.41,1520.41)(0.508,-0.564){23}{\rule{0.122pt}{1.107pt}}

\multiput(313.34,1522.70)(15.000,-14.703){2}{\rule{0.800pt}{0.553pt}}

\multiput(330.00,1506.09)(0.494,-0.507){25}{\rule{1.000pt}{0.122pt}}

\multiput(330.00,1506.34)(13.924,-16.000){2}{\rule{0.500pt}{0.800pt}}

\multiput(347.41,1487.63)(0.508,-0.529){23}{\rule{0.122pt}{1.053pt}}

\multiput(344.34,1489.81)(15.000,-13.814){2}{\rule{0.800pt}{0.527pt}}

\multiput(362.41,1471.41)(0.508,-0.564){23}{\rule{0.122pt}{1.107pt}}

\multiput(359.34,1473.70)(15.000,-14.703){2}{\rule{0.800pt}{0.553pt}}

\multiput(376.00,1457.09)(0.494,-0.507){25}{\rule{1.000pt}{0.122pt}}

\multiput(376.00,1457.34)(13.924,-16.000){2}{\rule{0.500pt}{0.800pt}}

\multiput(393.41,1438.63)(0.508,-0.529){23}{\rule{0.122pt}{1.053pt}}

\multiput(390.34,1440.81)(15.000,-13.814){2}{\rule{0.800pt}{0.527pt}}

\multiput(408.41,1422.64)(0.507,-0.527){25}{\rule{0.122pt}{1.050pt}}

\multiput(405.34,1424.82)(16.000,-14.821){2}{\rule{0.800pt}{0.525pt}}

\multiput(424.41,1405.63)(0.508,-0.529){23}{\rule{0.122pt}{1.053pt}}

\multiput(421.34,1407.81)(15.000,-13.814){2}{\rule{0.800pt}{0.527pt}}

\multiput(439.41,1389.64)(0.507,-0.527){25}{\rule{0.122pt}{1.050pt}}

\multiput(436.34,1391.82)(16.000,-14.821){2}{\rule{0.800pt}{0.525pt}}

\multiput(455.41,1372.63)(0.508,-0.529){23}{\rule{0.122pt}{1.053pt}}

\multiput(452.34,1374.81)(15.000,-13.814){2}{\rule{0.800pt}{0.527pt}}

\multiput(469.00,1359.09)(0.494,-0.507){25}{\rule{1.000pt}{0.122pt}}

\multiput(469.00,1359.34)(13.924,-16.000){2}{\rule{0.500pt}{0.800pt}}

\multiput(486.41,1340.41)(0.508,-0.564){23}{\rule{0.122pt}{1.107pt}}

\multiput(483.34,1342.70)(15.000,-14.703){2}{\rule{0.800pt}{0.553pt}}

\multiput(500.00,1326.09)(0.494,-0.507){25}{\rule{1.000pt}{0.122pt}}

\multiput(500.00,1326.34)(13.924,-16.000){2}{\rule{0.500pt}{0.800pt}}

\multiput(517.41,1307.63)(0.508,-0.529){23}{\rule{0.122pt}{1.053pt}}

\multiput(514.34,1309.81)(15.000,-13.814){2}{\rule{0.800pt}{0.527pt}}

\multiput(532.41,1291.41)(0.508,-0.564){23}{\rule{0.122pt}{1.107pt}}

\multiput(529.34,1293.70)(15.000,-14.703){2}{\rule{0.800pt}{0.553pt}}

\multiput(546.00,1277.09)(0.494,-0.507){25}{\rule{1.000pt}{0.122pt}}

\multiput(546.00,1277.34)(13.924,-16.000){2}{\rule{0.500pt}{0.800pt}}

\multiput(563.41,1258.41)(0.508,-0.564){23}{\rule{0.122pt}{1.107pt}}

\multiput(560.34,1260.70)(15.000,-14.703){2}{\rule{0.800pt}{0.553pt}}

\multiput(577.00,1244.09)(0.494,-0.507){25}{\rule{1.000pt}{0.122pt}}

\multiput(577.00,1244.34)(13.924,-16.000){2}{\rule{0.500pt}{0.800pt}}

\multiput(594.41,1225.63)(0.508,-0.529){23}{\rule{0.122pt}{1.053pt}}

\multiput(591.34,1227.81)(15.000,-13.814){2}{\rule{0.800pt}{0.527pt}}

\multiput(609.41,1209.64)(0.507,-0.527){25}{\rule{0.122pt}{1.050pt}}

\multiput(606.34,1211.82)(16.000,-14.821){2}{\rule{0.800pt}{0.525pt}}

\multiput(625.41,1192.63)(0.508,-0.529){23}{\rule{0.122pt}{1.053pt}}

\multiput(622.34,1194.81)(15.000,-13.814){2}{\rule{0.800pt}{0.527pt}}

\multiput(639.00,1179.09)(0.494,-0.507){25}{\rule{1.000pt}{0.122pt}}

\multiput(639.00,1179.34)(13.924,-16.000){2}{\rule{0.500pt}{0.800pt}}

\multiput(656.41,1160.41)(0.508,-0.564){23}{\rule{0.122pt}{1.107pt}}

\multiput(653.34,1162.70)(15.000,-14.703){2}{\rule{0.800pt}{0.553pt}}

\multiput(670.00,1146.09)(0.494,-0.507){25}{\rule{1.000pt}{0.122pt}}

\multiput(670.00,1146.34)(13.924,-16.000){2}{\rule{0.500pt}{0.800pt}}

\multiput(687.41,1127.41)(0.508,-0.564){23}{\rule{0.122pt}{1.107pt}}

\multiput(684.34,1129.70)(15.000,-14.703){2}{\rule{0.800pt}{0.553pt}}

\multiput(702.41,1110.63)(0.508,-0.529){23}{\rule{0.122pt}{1.053pt}}

\multiput(699.34,1112.81)(15.000,-13.814){2}{\rule{0.800pt}{0.527pt}}

\multiput(716.00,1097.09)(0.494,-0.507){25}{\rule{1.000pt}{0.122pt}}

\multiput(716.00,1097.34)(13.924,-16.000){2}{\rule{0.500pt}{0.800pt}}

\multiput(733.41,1078.41)(0.508,-0.564){23}{\rule{0.122pt}{1.107pt}}

\multiput(730.34,1080.70)(15.000,-14.703){2}{\rule{0.800pt}{0.553pt}}

\multiput(747.00,1064.09)(0.494,-0.507){25}{\rule{1.000pt}{0.122pt}}

\multiput(747.00,1064.34)(13.924,-16.000){2}{\rule{0.500pt}{0.800pt}}

\multiput(764.41,1045.63)(0.508,-0.529){23}{\rule{0.122pt}{1.053pt}}

\multiput(761.34,1047.81)(15.000,-13.814){2}{\rule{0.800pt}{0.527pt}}

\multiput(779.41,1029.64)(0.507,-0.527){25}{\rule{0.122pt}{1.050pt}}

\multiput(776.34,1031.82)(16.000,-14.821){2}{\rule{0.800pt}{0.525pt}}

\multiput(795.41,1012.63)(0.508,-0.529){23}{\rule{0.122pt}{1.053pt}}

\multiput(792.34,1014.81)(15.000,-13.814){2}{\rule{0.800pt}{0.527pt}}

\multiput(809.00,999.09)(0.494,-0.507){25}{\rule{1.000pt}{0.122pt}}

\multiput(809.00,999.34)(13.924,-16.000){2}{\rule{0.500pt}{0.800pt}}

\multiput(826.41,980.41)(0.508,-0.564){23}{\rule{0.122pt}{1.107pt}}

\multiput(823.34,982.70)(15.000,-14.703){2}{\rule{0.800pt}{0.553pt}}

\multiput(840.00,966.09)(0.494,-0.507){25}{\rule{1.000pt}{0.122pt}}

\multiput(840.00,966.34)(13.924,-16.000){2}{\rule{0.500pt}{0.800pt}}

\multiput(857.41,947.41)(0.508,-0.564){23}{\rule{0.122pt}{1.107pt}}

\multiput(854.34,949.70)(15.000,-14.703){2}{\rule{0.800pt}{0.553pt}}

\multiput(872.41,930.63)(0.508,-0.529){23}{\rule{0.122pt}{1.053pt}}

\multiput(869.34,932.81)(15.000,-13.814){2}{\rule{0.800pt}{0.527pt}}

\multiput(886.00,917.09)(0.494,-0.507){25}{\rule{1.000pt}{0.122pt}}

\multiput(886.00,917.34)(13.924,-16.000){2}{\rule{0.500pt}{0.800pt}}

\multiput(903.41,898.41)(0.508,-0.564){23}{\rule{0.122pt}{1.107pt}}

\multiput(900.34,900.70)(15.000,-14.703){2}{\rule{0.800pt}{0.553pt}}

\multiput(917.00,884.09)(0.494,-0.507){25}{\rule{1.000pt}{0.122pt}}

\multiput(917.00,884.34)(13.924,-16.000){2}{\rule{0.500pt}{0.800pt}}

\multiput(934.41,865.63)(0.508,-0.529){23}{\rule{0.122pt}{1.053pt}}

\multiput(931.34,867.81)(15.000,-13.814){2}{\rule{0.800pt}{0.527pt}}

\multiput(949.41,849.64)(0.507,-0.527){25}{\rule{0.122pt}{1.050pt}}

\multiput(946.34,851.82)(16.000,-14.821){2}{\rule{0.800pt}{0.525pt}}

\multiput(965.41,832.63)(0.508,-0.529){23}{\rule{0.122pt}{1.053pt}}

\multiput(962.34,834.81)(15.000,-13.814){2}{\rule{0.800pt}{0.527pt}}

\multiput(980.41,816.64)(0.507,-0.527){25}{\rule{0.122pt}{1.050pt}}

\multiput(977.34,818.82)(16.000,-14.821){2}{\rule{0.800pt}{0.525pt}}

\multiput(996.41,799.63)(0.508,-0.529){23}{\rule{0.122pt}{1.053pt}}

\multiput(993.34,801.81)(15.000,-13.814){2}{\rule{0.800pt}{0.527pt}}

\multiput(1010.00,786.09)(0.494,-0.507){25}{\rule{1.000pt}{0.122pt}}

\multiput(1010.00,786.34)(13.924,-16.000){2}{\rule{0.500pt}{0.800pt}}

\multiput(1027.41,767.41)(0.508,-0.564){23}{\rule{0.122pt}{1.107pt}}

\multiput(1024.34,769.70)(15.000,-14.703){2}{\rule{0.800pt}{0.553pt}}

\multiput(1042.41,750.63)(0.508,-0.529){23}{\rule{0.122pt}{1.053pt}}

\multiput(1039.34,752.81)(15.000,-13.814){2}{\rule{0.800pt}{0.527pt}}

\multiput(1056.00,737.09)(0.494,-0.507){25}{\rule{1.000pt}{0.122pt}}

\multiput(1056.00,737.34)(13.924,-16.000){2}{\rule{0.500pt}{0.800pt}}

\multiput(1073.41,718.41)(0.508,-0.564){23}{\rule{0.122pt}{1.107pt}}

\multiput(1070.34,720.70)(15.000,-14.703){2}{\rule{0.800pt}{0.553pt}}

\multiput(1087.00,704.09)(0.494,-0.507){25}{\rule{1.000pt}{0.122pt}}

\multiput(1087.00,704.34)(13.924,-16.000){2}{\rule{0.500pt}{0.800pt}}

\multiput(1104.41,685.63)(0.508,-0.529){23}{\rule{0.122pt}{1.053pt}}

\multiput(1101.34,687.81)(15.000,-13.814){2}{\rule{0.800pt}{0.527pt}}

\multiput(1119.41,669.64)(0.507,-0.527){25}{\rule{0.122pt}{1.050pt}}

\multiput(1116.34,671.82)(16.000,-14.821){2}{\rule{0.800pt}{0.525pt}}

\multiput(1135.41,652.63)(0.508,-0.529){23}{\rule{0.122pt}{1.053pt}}

\multiput(1132.34,654.81)(15.000,-13.814){2}{\rule{0.800pt}{0.527pt}}

\multiput(1150.41,636.64)(0.507,-0.527){25}{\rule{0.122pt}{1.050pt}}

\multiput(1147.34,638.82)(16.000,-14.821){2}{\rule{0.800pt}{0.525pt}}

\multiput(1166.41,619.63)(0.508,-0.529){23}{\rule{0.122pt}{1.053pt}}

\multiput(1163.34,621.81)(15.000,-13.814){2}{\rule{0.800pt}{0.527pt}}

\multiput(1180.00,606.09)(0.494,-0.507){25}{\rule{1.000pt}{0.122pt}}

\multiput(1180.00,606.34)(13.924,-16.000){2}{\rule{0.500pt}{0.800pt}}

\multiput(1197.41,587.41)(0.508,-0.564){23}{\rule{0.122pt}{1.107pt}}

\multiput(1194.34,589.70)(15.000,-14.703){2}{\rule{0.800pt}{0.553pt}}

\multiput(1212.41,570.63)(0.508,-0.529){23}{\rule{0.122pt}{1.053pt}}

\multiput(1209.34,572.81)(15.000,-13.814){2}{\rule{0.800pt}{0.527pt}}

\multiput(1226.00,557.09)(0.494,-0.507){25}{\rule{1.000pt}{0.122pt}}

\multiput(1226.00,557.34)(13.924,-16.000){2}{\rule{0.500pt}{0.800pt}}

\multiput(1243.41,538.41)(0.508,-0.564){23}{\rule{0.122pt}{1.107pt}}

\multiput(1240.34,540.70)(15.000,-14.703){2}{\rule{0.800pt}{0.553pt}}

\multiput(1257.00,524.09)(0.494,-0.507){25}{\rule{1.000pt}{0.122pt}}

\multiput(1257.00,524.34)(13.924,-16.000){2}{\rule{0.500pt}{0.800pt}}

\multiput(1274.41,505.41)(0.508,-0.564){23}{\rule{0.122pt}{1.107pt}}

\multiput(1271.34,507.70)(15.000,-14.703){2}{\rule{0.800pt}{0.553pt}}

\multiput(1288.00,491.09)(0.494,-0.507){25}{\rule{1.000pt}{0.122pt}}

\multiput(1288.00,491.34)(13.924,-16.000){2}{\rule{0.500pt}{0.800pt}}

\multiput(1305.41,472.63)(0.508,-0.529){23}{\rule{0.122pt}{1.053pt}}

\multiput(1302.34,474.81)(15.000,-13.814){2}{\rule{0.800pt}{0.527pt}}

\multiput(1320.41,456.64)(0.507,-0.527){25}{\rule{0.122pt}{1.050pt}}

\multiput(1317.34,458.82)(16.000,-14.821){2}{\rule{0.800pt}{0.525pt}}

\multiput(1336.41,439.63)(0.508,-0.529){23}{\rule{0.122pt}{1.053pt}}

\multiput(1333.34,441.81)(15.000,-13.814){2}{\rule{0.800pt}{0.527pt}}

\multiput(1350.00,426.09)(0.494,-0.507){25}{\rule{1.000pt}{0.122pt}}

\multiput(1350.00,426.34)(13.924,-16.000){2}{\rule{0.500pt}{0.800pt}}

\multiput(1367.41,407.41)(0.508,-0.564){23}{\rule{0.122pt}{1.107pt}}

\multiput(1364.34,409.70)(15.000,-14.703){2}{\rule{0.800pt}{0.553pt}}

\multiput(1382.41,390.63)(0.508,-0.529){23}{\rule{0.122pt}{1.053pt}}

\multiput(1379.34,392.81)(15.000,-13.814){2}{\rule{0.800pt}{0.527pt}}

\multiput(1397.41,374.64)(0.507,-0.527){25}{\rule{0.122pt}{1.050pt}}

\multiput(1394.34,376.82)(16.000,-14.821){2}{\rule{0.800pt}{0.525pt}}

\multiput(1413.41,357.63)(0.508,-0.529){23}{\rule{0.122pt}{1.053pt}}

\multiput(1410.34,359.81)(15.000,-13.814){2}{\rule{0.800pt}{0.527pt}}

\multiput(1427.00,344.09)(0.494,-0.507){25}{\rule{1.000pt}{0.122pt}}

\multiput(1427.00,344.34)(13.924,-16.000){2}{\rule{0.500pt}{0.800pt}}

\multiput(1444.41,325.41)(0.508,-0.564){23}{\rule{0.122pt}{1.107pt}}

\multiput(1441.34,327.70)(15.000,-14.703){2}{\rule{0.800pt}{0.553pt}}

\multiput(1458.00,311.09)(0.494,-0.507){25}{\rule{1.000pt}{0.122pt}}

\multiput(1458.00,311.34)(13.924,-16.000){2}{\rule{0.500pt}{0.800pt}}

\multiput(1475.41,292.63)(0.508,-0.529){23}{\rule{0.122pt}{1.053pt}}

\multiput(1472.34,294.81)(15.000,-13.814){2}{\rule{0.800pt}{0.527pt}}

\multiput(1490.41,276.64)(0.507,-0.527){25}{\rule{0.122pt}{1.050pt}}

\multiput(1487.34,278.82)(16.000,-14.821){2}{\rule{0.800pt}{0.525pt}}

\multiput(1506.41,259.63)(0.508,-0.529){23}{\rule{0.122pt}{1.053pt}}

\multiput(1503.34,261.81)(15.000,-13.814){2}{\rule{0.800pt}{0.527pt}}

\multiput(1520.00,246.09)(0.494,-0.507){25}{\rule{1.000pt}{0.122pt}}

\multiput(1520.00,246.34)(13.924,-16.000){2}{\rule{0.500pt}{0.800pt}}

\multiput(1537.41,227.41)(0.508,-0.564){23}{\rule{0.122pt}{1.107pt}}

\multiput(1534.34,229.70)(15.000,-14.703){2}{\rule{0.800pt}{0.553pt}}

\multiput(1552.41,210.63)(0.508,-0.529){23}{\rule{0.122pt}{1.053pt}}

\multiput(1549.34,212.81)(15.000,-13.814){2}{\rule{0.800pt}{0.527pt}}

\multiput(1567.41,194.64)(0.507,-0.527){25}{\rule{0.122pt}{1.050pt}}

\multiput(1564.34,196.82)(16.000,-14.821){2}{\rule{0.800pt}{0.525pt}}

\multiput(1583.41,177.63)(0.508,-0.529){23}{\rule{0.122pt}{1.053pt}}

\multiput(1580.34,179.81)(15.000,-13.814){2}{\rule{0.800pt}{0.527pt}}

\multiput(1597.00,164.09)(0.494,-0.507){25}{\rule{1.000pt}{0.122pt}}

\multiput(1597.00,164.34)(13.924,-16.000){2}{\rule{0.500pt}{0.800pt}}

\multiput(1614.41,145.41)(0.508,-0.564){23}{\rule{0.122pt}{1.107pt}}

\multiput(1611.34,147.70)(15.000,-14.703){2}{\rule{0.800pt}{0.553pt}}

\put(1628,130.34){\rule{0.482pt}{0.800pt}}

\multiput(1628.00,131.34)(1.000,-2.000){2}{\rule{0.241pt}{0.800pt}}

\sbox{\plotpoint}{\rule[-0.200pt]{0.400pt}{0.400pt}}%

\put(191.0,131.0){\rule[-0.200pt]{0.400pt}{385.440pt}}

\put(191.0,131.0){\rule[-0.200pt]{368.577pt}{0.400pt}}

\put(1721.0,131.0){\rule[-0.200pt]{0.400pt}{385.440pt}}

\put(191.0,1731.0){\rule[-0.200pt]{368.577pt}{0.400pt}}

\end{picture}
    \end{center}
  \end{figure}

Az illesztett egyenes tengelymetszetei 9,41 (x tengely) és 9,53 (y tengely), átlaguk, ezek reciproka pedig 0,1063 és 0,1049 (m). Tehát ezek a fókusztávolságra kapott értékek. A hiteles érték 0,10 m.



\section{Szórólencse képalkotása}


A fényforrástól $t=20\unit{cm}$-re elhelyezünk egy szórólencsét. Ha belenézünk, kicsinyített egyenes képet látunk, a tárgytávolságnál közelebb. A szórólencse képe virtuális, ernyővel nem tudjuk felfogni. Viszont egy gyűjtőlencsével az így kapott virtuális képet tárgyként felhasználva valódi képet készíthetünk róla.

Elhelyezzük tehát a gyűjtőlencsét a szórólencse mögé, és emögé egy ernyőt teszünk. Az ernyő és a gyűjtőlencse közötti távolság $k' = 0,380$ m, a  gyűjtőlencse és a szórólencse  közötti távolság $d = 0,300$ m, és így éles a kép.  


Ha eltávolítjuk a szórólencsét, a kép újra homályossá válik. Ahhoz, hogy éles legyen, a fényforrást közelebb kell húznunk a gyűjtőlencséhez, $t' = 0,391$ m távolságra.  Ebből tehát tudjuk, hogy a szórólencse virtuális képe is ebben a pozícióban kellett, hogy legyen ahhoz, hogy éles képet alkosson róla a gyűjtőlencse. Eszerint a szórólencse virtuális képe $k = t'-d = 0,091$ m-re volt a szórólencsétől. Eszerint a szórólencse fókusztávolsága $f = 1/(1/t-1/k) = -0,167$ m (névlegesen -0,150 m). A szórólencse nagyítása $k/t=0,46$.


\section{Elhajlás résen}

Ha a fény résen halad keresztül, elhajlás (diffrakció) jön létre. Az ernyőn a fény foltokban jelenik meg:

\vskip 14mm

A minimumhelyekre az \[ a \sin \theta = n \lambda\] összefüggés teljesül, ahol $a$ a rés szélessége, $\theta$ a minimumhely szöge és $n$ pozitív egész szám, $\lambda$ a hullámhossz. Mivel a szögek kicsik, $\sin\theta$ közelíthető $\tan  \theta$-val, vagyis a minimumhely középponttól való távolságának és a rés-ernyő távolságnak hányadosával $(y/D)$.

Különböző réseknél megmérjük a két első ($n=1$) minimumhely közti távolságot, ill.\ a két második minimumhely közöttit is. A rés-ernyő távolság $D=1108$ mm. 

  
  \begin{table}[H]
    \begin{center}
      \begin{tabular}{|
c|
c|
c|
}
        \hline
        
rés (mm) & 
\ensuremath{\unit[\text{\ensuremath{d_{min,1}}}]{(mm)}} & \ensuremath{\unit[\text{\ensuremath{d_{min,2}}}]{(mm)}}
\\
        \hline\hline
        
0.04
 & 38
 & 78
\\
        \hline
        
0.08
 & 19
 & 39
\\
        \hline
        
0.16
 & 9
 & 18
\\
        \hline
      \end{tabular}
      \caption{A mért adatok}
      \label{tab:}
    \end{center}
  \end{table}

Kiszámoljuk a távolságot a középponttól ($d_{min}$ fele), majd a hullámhossz (670 nm) ismeretében kiszámoljuk a rés méretét az \[a=(n\lambda D )/ y \] képlet alapján, és ezt összevetjük az eredetivel. Hibaszámítást is végzünk, a szorzás és osztás hibaterjedési képletei alapján.

  
  \begin{table}[H]
    \begin{center}
      \begin{tabular}{|
c|
c|
c|
c|
c|
c|
c|
}
        \hline
        
$a$ (mm) & 
\ensuremath{\unit[\text{\ensuremath{y_{1}}}]{(mm)}} & \ensuremath{\unit[\text{\ensuremath{a_{1}}}]{(mm)}} & \ensuremath{\delta{}a_{1}} & \ensuremath{\unit[\text{\ensuremath{y_{2}}}]{(mm)}} & \ensuremath{\unit[\text{\ensuremath{a_{2}}}]{(mm)}} & \ensuremath{\delta{}a_{2}}
\\
        \hline\hline
        
0.04
 & 19,0
 & 0,039
 & 0,01
 & 39,0
 & 0,0381
 & 0,008
\\
        \hline
        
0.08
 & 9,5
 & 0,078
 & 0,03
 & 19,5
 & 0,076
 & 0,01
\\
        \hline
        
0.16
 & 4,5
 & 0,16
 & 0,06
 & 9,0
 & 0,165
 & 0,03
\\
        \hline
      \end{tabular}
      \caption{Kiértékelés}
      \label{tab:}
    \end{center}
  \end{table}


Láthatjuk hogy a kapott értékek ($a_1,a_2$) többnyire a hibahatáron belül megközelítik az hitelesített értékeket ($a$).

Láthatjuk továbbá, hogy ahogy növeljük a rést, az elhajlás mértéke, és ezzel a minimumok távolsága is, csökken.

\end{document}

