\documentclass[12pt]{article}
\usepackage[utf8]{inputenc}
\usepackage[final]{graphics}
\usepackage[pdftex]{graphicx}
\usepackage[dvips]{color}
\usepackage{amsfonts}
\usepackage{subfigure}
\usepackage{lscape}
\usepackage{hyperref}
\usepackage{amsmath}
\usepackage{units}
\usepackage{float}
\usepackage[table]{xcolor}
\usepackage{rmpage}
\usepackage[magyar]{babel}
\usepackage[margin=2cm]{geometry}
\newcommand{\hide}[1]{}
\DeclareGraphicsExtensions{.jpg,.pdf,.mps,.png}
    \title{Állóhullám kötélen}
    \author{Kalló Bernát -- Mérés: 2012.04.25. -- Leadás: 2012.05.02.}
    \date{}
    \begin{document}
\def\lambdan{\ensuremath{\lambda_n}}\def\fmin{\ensuremath{f_\min}}\def\fix{\ensuremath{f_i}}\def\fmax{\ensuremath{f_\max}}\def\Deltaf{\ensuremath{\Delta{}f}}
\maketitle

\section{Különböző módusok frekvenciái}

\paragraph{A mérés célja}

Kötélen vizsgálunk állóhullámokat. A kötél egyik végét egy súly feszíti, a másik végén egy állítható frekvenciájú szinuszos rezgéskeltő van.

Az első kísérletünkben állandó súllyal különböző állóhullám-módusokat állítunk be a rezgéskeltő segítségével. Azt szeretnénk belátni, hogy az $f$ frekvencia és a rezgő szakaszok $n$ száma között a \[f=\frac{c}{\lambdan}=\frac{cn}{2L}\] összefüggés áll fenn, ahol $c$ a hullám terjedési sebessége (független $n$-től), $L$ a rezgő kötél hossza, és $\lambdan=\frac{2L}{n}$ az $n$-edik módushoz tartozó hullámhossz (a rezgő szakaszok hosszának duplája).

Megmértem a kötél hosszát is: $L=1,43$ m.

\paragraph{A mérés menete}

$n=2,\dots,6$ szakaszból álló módusokat állítok be a rezgéskeltő frekvenciájának állításával. Minden módusra megnézem, hogy mennyi a legkisebb ill.\ a legnagyobb érték (\fmin és \fmax), amire még elég nagy amplitúdóval látszik az állóhullám.


\paragraph{Kiértékelés}

\fmin{} és \fmax{} átlagával fogok számolni. Hibabecsléshez a különbségük (\Deltaf) átlagát fogom venni, ami 0,2.

  
  \begin{table}[H]
    \begin{center}
      \begin{tabular}{|
c|
c|
c|
c|
c|
}
        \hline
        
\ensuremath{n} & 
\ensuremath{\unit[\text{\ensuremath{f_\min}}]{(Hz)}} & \ensuremath{\unit[\text{\ensuremath{f_\max}}]{(Hz)}} & \ensuremath{\unit[\text{\ensuremath{\Delta{}f}}]{(Hz)}} & \ensuremath{\unit[\text{\ensuremath{f}}]{(Hz)}}
\\
        \hline\hline
        
2
 & 13,6
 & 13,6
 & 0
 & 13,6 \pm 0,20
\\
        \hline
        
3
 & 20,4
 & 20,6
 & 0,2
 & 20,5 \pm 0,20
\\
        \hline
        
4
 & 27,4
 & 27,6
 & 0,2
 & 27,5 \pm 0,20
\\
        \hline
        
5
 & 34,3
 & 34,3
 & 0
 & 34,3 \pm 0,20
\\
        \hline
        
6
 & 41,1
 & 41,7
 & 0,6
 & 41,4 \pm 0,20
\\
        \hline
      \end{tabular}
      \caption{A mért adatok és számolt értékek}
      \label{tab:}
    \end{center}
  \end{table}

Ezután ábrázolom grafikonon $f$-et $n$ függvényében:

  \begin{figure}[H]
    \begin{center}
% GNUPLOT: LaTeX picture

\setlength{\unitlength}{0.240900pt}

\ifx\plotpoint\undefined\newsavebox{\plotpoint}\fi

\sbox{\plotpoint}{\rule[-0.200pt]{0.400pt}{0.400pt}}%

\begin{picture}(2007,1181)(0,0)

\sbox{\plotpoint}{\rule[-0.200pt]{0.400pt}{0.400pt}}%

\put(191.0,131.0){\rule[-0.200pt]{4.818pt}{0.400pt}}

\put(171,131){\makebox(0,0)[r]{ 0}}

\put(1937.0,131.0){\rule[-0.200pt]{4.818pt}{0.400pt}}

\put(191.0,243.0){\rule[-0.200pt]{4.818pt}{0.400pt}}

\put(171,243){\makebox(0,0)[r]{ 5}}

\put(1937.0,243.0){\rule[-0.200pt]{4.818pt}{0.400pt}}

\put(191.0,355.0){\rule[-0.200pt]{4.818pt}{0.400pt}}

\put(171,355){\makebox(0,0)[r]{ 10}}

\put(1937.0,355.0){\rule[-0.200pt]{4.818pt}{0.400pt}}

\put(191.0,468.0){\rule[-0.200pt]{4.818pt}{0.400pt}}

\put(171,468){\makebox(0,0)[r]{ 15}}

\put(1937.0,468.0){\rule[-0.200pt]{4.818pt}{0.400pt}}

\put(191.0,580.0){\rule[-0.200pt]{4.818pt}{0.400pt}}

\put(171,580){\makebox(0,0)[r]{ 20}}

\put(1937.0,580.0){\rule[-0.200pt]{4.818pt}{0.400pt}}

\put(191.0,692.0){\rule[-0.200pt]{4.818pt}{0.400pt}}

\put(171,692){\makebox(0,0)[r]{ 25}}

\put(1937.0,692.0){\rule[-0.200pt]{4.818pt}{0.400pt}}

\put(191.0,804.0){\rule[-0.200pt]{4.818pt}{0.400pt}}

\put(171,804){\makebox(0,0)[r]{ 30}}

\put(1937.0,804.0){\rule[-0.200pt]{4.818pt}{0.400pt}}

\put(191.0,917.0){\rule[-0.200pt]{4.818pt}{0.400pt}}

\put(171,917){\makebox(0,0)[r]{ 35}}

\put(1937.0,917.0){\rule[-0.200pt]{4.818pt}{0.400pt}}

\put(191.0,1029.0){\rule[-0.200pt]{4.818pt}{0.400pt}}

\put(171,1029){\makebox(0,0)[r]{ 40}}

\put(1937.0,1029.0){\rule[-0.200pt]{4.818pt}{0.400pt}}

\put(191.0,1141.0){\rule[-0.200pt]{4.818pt}{0.400pt}}

\put(171,1141){\makebox(0,0)[r]{ 45}}

\put(1937.0,1141.0){\rule[-0.200pt]{4.818pt}{0.400pt}}

\put(191.0,131.0){\rule[-0.200pt]{0.400pt}{4.818pt}}

\put(191,90){\makebox(0,0){ 0}}

\put(191.0,1121.0){\rule[-0.200pt]{0.400pt}{4.818pt}}

\put(485.0,131.0){\rule[-0.200pt]{0.400pt}{4.818pt}}

\put(485,90){\makebox(0,0){ 1}}

\put(485.0,1121.0){\rule[-0.200pt]{0.400pt}{4.818pt}}

\put(780.0,131.0){\rule[-0.200pt]{0.400pt}{4.818pt}}

\put(780,90){\makebox(0,0){ 2}}

\put(780.0,1121.0){\rule[-0.200pt]{0.400pt}{4.818pt}}

\put(1074.0,131.0){\rule[-0.200pt]{0.400pt}{4.818pt}}

\put(1074,90){\makebox(0,0){ 3}}

\put(1074.0,1121.0){\rule[-0.200pt]{0.400pt}{4.818pt}}

\put(1368.0,131.0){\rule[-0.200pt]{0.400pt}{4.818pt}}

\put(1368,90){\makebox(0,0){ 4}}

\put(1368.0,1121.0){\rule[-0.200pt]{0.400pt}{4.818pt}}

\put(1663.0,131.0){\rule[-0.200pt]{0.400pt}{4.818pt}}

\put(1663,90){\makebox(0,0){ 5}}

\put(1663.0,1121.0){\rule[-0.200pt]{0.400pt}{4.818pt}}

\put(1957.0,131.0){\rule[-0.200pt]{0.400pt}{4.818pt}}

\put(1957,90){\makebox(0,0){ 6}}

\put(1957.0,1121.0){\rule[-0.200pt]{0.400pt}{4.818pt}}

\put(191.0,131.0){\rule[-0.200pt]{0.400pt}{243.309pt}}

\put(191.0,131.0){\rule[-0.200pt]{425.429pt}{0.400pt}}

\put(1957.0,131.0){\rule[-0.200pt]{0.400pt}{243.309pt}}

\put(191.0,1141.0){\rule[-0.200pt]{425.429pt}{0.400pt}}

\put(70,636){\makebox(0,0){\ensuremath{\unit[\text{\ensuremath{f}}]{(Hz)}}}}

\put(1074,29){\makebox(0,0){\ensuremath{n}}}

\put(780.0,432.0){\rule[-0.200pt]{0.400pt}{2.168pt}}

\put(770.0,432.0){\rule[-0.200pt]{4.818pt}{0.400pt}}

\put(770.0,441.0){\rule[-0.200pt]{4.818pt}{0.400pt}}

\put(1074.0,587.0){\rule[-0.200pt]{0.400pt}{2.168pt}}

\put(1064.0,587.0){\rule[-0.200pt]{4.818pt}{0.400pt}}

\put(1064.0,596.0){\rule[-0.200pt]{4.818pt}{0.400pt}}

\put(1368.0,744.0){\rule[-0.200pt]{0.400pt}{2.168pt}}

\put(1358.0,744.0){\rule[-0.200pt]{4.818pt}{0.400pt}}

\put(1358.0,753.0){\rule[-0.200pt]{4.818pt}{0.400pt}}

\put(1663.0,896.0){\rule[-0.200pt]{0.400pt}{2.168pt}}

\put(1653.0,896.0){\rule[-0.200pt]{4.818pt}{0.400pt}}

\put(1653.0,905.0){\rule[-0.200pt]{4.818pt}{0.400pt}}

\put(1957.0,1056.0){\rule[-0.200pt]{0.400pt}{2.168pt}}

\put(1947.0,1056.0){\rule[-0.200pt]{4.818pt}{0.400pt}}

\put(780,436){\raisebox{-.8pt}{\makebox(0,0){$\Diamond$}}}

\put(1074,591){\raisebox{-.8pt}{\makebox(0,0){$\Diamond$}}}

\put(1368,748){\raisebox{-.8pt}{\makebox(0,0){$\Diamond$}}}

\put(1663,901){\raisebox{-.8pt}{\makebox(0,0){$\Diamond$}}}

\put(1957,1060){\raisebox{-.8pt}{\makebox(0,0){$\Diamond$}}}

\put(1947.0,1065.0){\rule[-0.200pt]{4.818pt}{0.400pt}}

\put(204.00,131.00){\usebox{\plotpoint}}

\put(222.35,140.67){\usebox{\plotpoint}}

\put(240.60,150.55){\usebox{\plotpoint}}

\put(258.89,160.36){\usebox{\plotpoint}}

\put(277.42,169.71){\usebox{\plotpoint}}

\put(295.62,179.68){\usebox{\plotpoint}}

\put(314.13,189.07){\usebox{\plotpoint}}

\put(332.32,199.06){\usebox{\plotpoint}}

\put(350.84,208.42){\usebox{\plotpoint}}

\put(368.77,218.87){\usebox{\plotpoint}}

\put(387.32,228.18){\usebox{\plotpoint}}

\put(405.48,238.24){\usebox{\plotpoint}}

\put(424.04,247.52){\usebox{\plotpoint}}

\put(442.57,256.87){\usebox{\plotpoint}}

\put(460.73,266.92){\usebox{\plotpoint}}

\put(479.04,276.69){\usebox{\plotpoint}}

\put(497.26,286.63){\usebox{\plotpoint}}

\put(515.74,296.08){\usebox{\plotpoint}}

\put(533.97,305.98){\usebox{\plotpoint}}

\put(552.43,315.46){\usebox{\plotpoint}}

\put(570.63,325.45){\usebox{\plotpoint}}

\put(589.04,335.02){\usebox{\plotpoint}}

\put(607.46,344.59){\usebox{\plotpoint}}

\put(625.75,354.38){\usebox{\plotpoint}}

\put(644.15,363.97){\usebox{\plotpoint}}

\put(662.46,373.73){\usebox{\plotpoint}}

\put(680.74,383.55){\usebox{\plotpoint}}

\put(698.95,393.48){\usebox{\plotpoint}}

\put(717.52,402.76){\usebox{\plotpoint}}

\put(735.85,412.47){\usebox{\plotpoint}}

\put(754.23,422.11){\usebox{\plotpoint}}

\put(772.55,431.86){\usebox{\plotpoint}}

\put(790.94,441.47){\usebox{\plotpoint}}

\put(809.08,451.52){\usebox{\plotpoint}}

\put(827.43,461.21){\usebox{\plotpoint}}

\put(845.70,471.05){\usebox{\plotpoint}}

\put(864.14,480.57){\usebox{\plotpoint}}

\put(882.70,489.85){\usebox{\plotpoint}}

\put(900.94,499.75){\usebox{\plotpoint}}

\put(919.24,509.54){\usebox{\plotpoint}}

\put(937.41,519.56){\usebox{\plotpoint}}

\put(955.92,528.96){\usebox{\plotpoint}}

\put(974.11,538.95){\usebox{\plotpoint}}

\put(992.63,548.31){\usebox{\plotpoint}}

\put(1010.80,558.34){\usebox{\plotpoint}}

\put(1029.14,568.07){\usebox{\plotpoint}}

\put(1047.68,577.38){\usebox{\plotpoint}}

\put(1065.85,587.42){\usebox{\plotpoint}}

\put(1084.38,596.77){\usebox{\plotpoint}}

\put(1102.56,606.78){\usebox{\plotpoint}}

\put(1121.05,616.20){\usebox{\plotpoint}}

\put(1139.05,626.52){\usebox{\plotpoint}}

\put(1157.61,635.81){\usebox{\plotpoint}}

\put(1176.08,645.27){\usebox{\plotpoint}}

\put(1194.32,655.16){\usebox{\plotpoint}}

\put(1212.78,664.65){\usebox{\plotpoint}}

\put(1230.97,674.63){\usebox{\plotpoint}}

\put(1249.25,684.47){\usebox{\plotpoint}}

\put(1267.54,694.27){\usebox{\plotpoint}}

\put(1285.94,703.86){\usebox{\plotpoint}}

\put(1304.25,713.63){\usebox{\plotpoint}}

\put(1322.81,722.91){\usebox{\plotpoint}}

\put(1341.07,732.75){\usebox{\plotpoint}}

\put(1359.30,742.65){\usebox{\plotpoint}}

\put(1377.64,752.36){\usebox{\plotpoint}}

\put(1396.01,762.01){\usebox{\plotpoint}}

\put(1414.34,771.74){\usebox{\plotpoint}}

\put(1432.72,781.36){\usebox{\plotpoint}}

\put(1451.03,791.13){\usebox{\plotpoint}}

\put(1469.30,800.98){\usebox{\plotpoint}}

\put(1487.80,810.40){\usebox{\plotpoint}}

\put(1506.06,820.25){\usebox{\plotpoint}}

\put(1524.51,829.75){\usebox{\plotpoint}}

\put(1542.75,839.64){\usebox{\plotpoint}}

\put(1561.22,849.11){\usebox{\plotpoint}}

\put(1579.25,859.38){\usebox{\plotpoint}}

\put(1597.71,868.85){\usebox{\plotpoint}}

\put(1616.27,878.13){\usebox{\plotpoint}}

\put(1634.45,888.14){\usebox{\plotpoint}}

\put(1652.98,897.49){\usebox{\plotpoint}}

\put(1671.15,907.53){\usebox{\plotpoint}}

\put(1689.48,917.26){\usebox{\plotpoint}}

\put(1707.63,927.32){\usebox{\plotpoint}}

\put(1726.17,936.65){\usebox{\plotpoint}}

\put(1744.35,946.67){\usebox{\plotpoint}}

\put(1762.91,955.96){\usebox{\plotpoint}}

\put(1781.38,965.40){\usebox{\plotpoint}}

\put(1799.40,975.70){\usebox{\plotpoint}}

\put(1817.87,985.15){\usebox{\plotpoint}}

\put(1836.11,995.05){\usebox{\plotpoint}}

\put(1854.57,1004.54){\usebox{\plotpoint}}

\put(1872.82,1014.41){\usebox{\plotpoint}}

\put(1891.19,1024.05){\usebox{\plotpoint}}

\put(1909.31,1034.15){\usebox{\plotpoint}}

\put(1927.87,1043.44){\usebox{\plotpoint}}

\put(1946.27,1053.04){\usebox{\plotpoint}}

\put(1957,1059){\usebox{\plotpoint}}

\put(191.0,131.0){\rule[-0.200pt]{0.400pt}{243.309pt}}

\put(191.0,131.0){\rule[-0.200pt]{425.429pt}{0.400pt}}

\put(1957.0,131.0){\rule[-0.200pt]{0.400pt}{243.309pt}}

\put(191.0,1141.0){\rule[-0.200pt]{425.429pt}{0.400pt}}

\end{picture}
    \end{center}
\caption{A frkvencia és a módusszám közti összefüggés}  \end{figure}

A GNUPLOT-tal illesztett egyenes egyenlete $f=-0,3+6,94*n$, korrelációja 0,943, tehát jó közelítéssel origón átmenő egyenest kaptunk. Ez azt jelenti, hogy igazoltuk az $f=\frac{c}{2L}\cdot n$ összefüggést, ebből $c=\unit[6,94]{Hz}\cdot 2L=\unitfrac[19,8]{m}{s}$.


\section{Különböző húzóerők}

\paragraph{A mérés célja}

A következő kísérletben egy másik kötéllel különböző húzóerőknél megmérjük egy állóhullám frekvenciáját. Ebből ki fogjuk számolni a hullám terjedési sebességét, abból pedig a szál lineáris sűrűségét.

\paragraph{A mérés menete}

Különböző húzóerők mellett úgy állítjuk be a frekvenciát, hogy 3 szakaszos állóhullámok alakuljanak ki ($n$=3). Minden mérésnél úgy próbáljuk beállítani, hogy a maximális amplitúdót érjük el. Három mérést végzünk, és az átlagukat vesszük $f$-nek, a szórásuk háromszorosát pedig hibakorlátnak.


  
  \begin{table}[H]
    \begin{center}
      \begin{tabular}{|
c|
c|
c|
c|
c|
}
        \hline
        
~ & 
\multicolumn{3}{c|}{\ensuremath{\unit[\text{\ensuremath{f_i}}]{(Hz)}}} & \raisebox{-1.0\totalheight}[1ex][1ex]{\ensuremath{\unit[\text{\ensuremath{f}}]{(Hz)}}}
\\\cline{2-4}
        
\ensuremath{\unit[\text{\ensuremath{m}}]{(g)}} & 
1 & 2 & 3 & ~
\\
        \hline\hline
        
50
 & 82,5
 & 83,0
 & 82,5
 & $83 \pm 0,9$
\\
        \hline
        
70
 & 97,2
 & 97,2
 & 97,7
 & $97 \pm 0,9$
\\
        \hline
        
90
 & 110,7
 & 110,9
 & 110,6
 & $110,7 \pm 0,46$
\\
        \hline
        
110
 & 122,3
 & 122,4
 & 122,2
 & $122,3 \pm 0,30$
\\
        \hline
        
130
 & 133,2
 & 133,0
 & 133,1
 & $133,1 \pm 0,30$
\\
        \hline
        
150
 & 143,1
 & 143,1
 & 143,0
 & $143,1 \pm 0,18$
\\
        \hline
        
170
 & 152,5
 & 152,5
 & 152,4
 & $152,5 \pm 0,18$
\\
        \hline
      \end{tabular}
      \caption{A mért adatok}
      \label{tab:}
    \end{center}
  \end{table}

\paragraph{Kiértékelés}

A kötélen a terjedési sebesség függ a feszítőerőtől ($F$) és a kötél lineáris sűrűségétől ($\mu$):

\[ v = \sqrt{\frac{F}{\mu}} \]

Mivel $F = mg$, $v=f\lambda$ és $\lambda=\frac{2L}{n}$, a frekvenciára a \[ f^2=\frac{n^2g}{4L^2\mu}m = a\cdot m\] kifejezést kapjuk.

Ábrázoljuk grafikonon $f^2$-et $m$ szerint, és illesszünk rá egyenest:

  \begin{figure}[H]
    \begin{center}
% GNUPLOT: LaTeX picture

\setlength{\unitlength}{0.240900pt}

\ifx\plotpoint\undefined\newsavebox{\plotpoint}\fi

\sbox{\plotpoint}{\rule[-0.200pt]{0.400pt}{0.400pt}}%

\begin{picture}(2007,1062)(0,0)

\sbox{\plotpoint}{\rule[-0.200pt]{0.400pt}{0.400pt}}%

\put(251.0,131.0){\rule[-0.200pt]{4.818pt}{0.400pt}}

\put(231,131){\makebox(0,0)[r]{ 0}}

\put(1937.0,131.0){\rule[-0.200pt]{4.818pt}{0.400pt}}

\put(251.0,309.0){\rule[-0.200pt]{4.818pt}{0.400pt}}

\put(231,309){\makebox(0,0)[r]{ 5000}}

\put(1937.0,309.0){\rule[-0.200pt]{4.818pt}{0.400pt}}

\put(251.0,487.0){\rule[-0.200pt]{4.818pt}{0.400pt}}

\put(231,487){\makebox(0,0)[r]{ 10000}}

\put(1937.0,487.0){\rule[-0.200pt]{4.818pt}{0.400pt}}

\put(251.0,666.0){\rule[-0.200pt]{4.818pt}{0.400pt}}

\put(231,666){\makebox(0,0)[r]{ 15000}}

\put(1937.0,666.0){\rule[-0.200pt]{4.818pt}{0.400pt}}

\put(251.0,844.0){\rule[-0.200pt]{4.818pt}{0.400pt}}

\put(231,844){\makebox(0,0)[r]{ 20000}}

\put(1937.0,844.0){\rule[-0.200pt]{4.818pt}{0.400pt}}

\put(251.0,1022.0){\rule[-0.200pt]{4.818pt}{0.400pt}}

\put(231,1022){\makebox(0,0)[r]{ 25000}}

\put(1937.0,1022.0){\rule[-0.200pt]{4.818pt}{0.400pt}}

\put(251.0,131.0){\rule[-0.200pt]{0.400pt}{4.818pt}}

\put(251,90){\makebox(0,0){ 0}}

\put(251.0,1002.0){\rule[-0.200pt]{0.400pt}{4.818pt}}

\put(441.0,131.0){\rule[-0.200pt]{0.400pt}{4.818pt}}

\put(441,90){\makebox(0,0){ 20}}

\put(441.0,1002.0){\rule[-0.200pt]{0.400pt}{4.818pt}}

\put(630.0,131.0){\rule[-0.200pt]{0.400pt}{4.818pt}}

\put(630,90){\makebox(0,0){ 40}}

\put(630.0,1002.0){\rule[-0.200pt]{0.400pt}{4.818pt}}

\put(820.0,131.0){\rule[-0.200pt]{0.400pt}{4.818pt}}

\put(820,90){\makebox(0,0){ 60}}

\put(820.0,1002.0){\rule[-0.200pt]{0.400pt}{4.818pt}}

\put(1009.0,131.0){\rule[-0.200pt]{0.400pt}{4.818pt}}

\put(1009,90){\makebox(0,0){ 80}}

\put(1009.0,1002.0){\rule[-0.200pt]{0.400pt}{4.818pt}}

\put(1199.0,131.0){\rule[-0.200pt]{0.400pt}{4.818pt}}

\put(1199,90){\makebox(0,0){ 100}}

\put(1199.0,1002.0){\rule[-0.200pt]{0.400pt}{4.818pt}}

\put(1388.0,131.0){\rule[-0.200pt]{0.400pt}{4.818pt}}

\put(1388,90){\makebox(0,0){ 120}}

\put(1388.0,1002.0){\rule[-0.200pt]{0.400pt}{4.818pt}}

\put(1578.0,131.0){\rule[-0.200pt]{0.400pt}{4.818pt}}

\put(1578,90){\makebox(0,0){ 140}}

\put(1578.0,1002.0){\rule[-0.200pt]{0.400pt}{4.818pt}}

\put(1767.0,131.0){\rule[-0.200pt]{0.400pt}{4.818pt}}

\put(1767,90){\makebox(0,0){ 160}}

\put(1767.0,1002.0){\rule[-0.200pt]{0.400pt}{4.818pt}}

\put(1957.0,131.0){\rule[-0.200pt]{0.400pt}{4.818pt}}

\put(1957,90){\makebox(0,0){ 180}}

\put(1957.0,1002.0){\rule[-0.200pt]{0.400pt}{4.818pt}}

\put(251.0,131.0){\rule[-0.200pt]{0.400pt}{214.642pt}}

\put(251.0,131.0){\rule[-0.200pt]{410.975pt}{0.400pt}}

\put(1957.0,131.0){\rule[-0.200pt]{0.400pt}{214.642pt}}

\put(251.0,1022.0){\rule[-0.200pt]{410.975pt}{0.400pt}}

\put(70,576){\makebox(0,0){$\unit[f^2]{(Hz^2)}$}}

\put(1104,29){\makebox(0,0){$m$ (g)}}

\put(725,375){\raisebox{-.8pt}{\makebox(0,0){$\Diamond$}}}

\put(914,469){\raisebox{-.8pt}{\makebox(0,0){$\Diamond$}}}

\put(1104,568){\raisebox{-.8pt}{\makebox(0,0){$\Diamond$}}}

\put(1294,664){\raisebox{-.8pt}{\makebox(0,0){$\Diamond$}}}

\put(1483,762){\raisebox{-.8pt}{\makebox(0,0){$\Diamond$}}}

\put(1673,860){\raisebox{-.8pt}{\makebox(0,0){$\Diamond$}}}

\put(1862,959){\raisebox{-.8pt}{\makebox(0,0){$\Diamond$}}}

\put(255.00,131.00){\usebox{\plotpoint}}

\put(273.64,140.12){\usebox{\plotpoint}}

\put(292.11,149.56){\usebox{\plotpoint}}

\put(310.47,159.23){\usebox{\plotpoint}}

\put(328.70,169.14){\usebox{\plotpoint}}

\put(347.35,178.23){\usebox{\plotpoint}}

\put(365.91,187.51){\usebox{\plotpoint}}

\put(384.08,197.54){\usebox{\plotpoint}}

\put(402.71,206.69){\usebox{\plotpoint}}

\put(421.22,216.06){\usebox{\plotpoint}}

\put(439.55,225.78){\usebox{\plotpoint}}

\put(457.97,235.34){\usebox{\plotpoint}}

\put(476.47,244.74){\usebox{\plotpoint}}

\put(495.04,254.02){\usebox{\plotpoint}}

\put(513.21,264.04){\usebox{\plotpoint}}

\put(531.84,273.16){\usebox{\plotpoint}}

\put(550.09,283.04){\usebox{\plotpoint}}

\put(568.65,292.33){\usebox{\plotpoint}}

\put(587.08,301.87){\usebox{\plotpoint}}

\put(605.58,311.29){\usebox{\plotpoint}}

\put(624.14,320.57){\usebox{\plotpoint}}

\put(642.50,330.25){\usebox{\plotpoint}}

\put(660.99,339.68){\usebox{\plotpoint}}

\put(679.21,349.60){\usebox{\plotpoint}}

\put(697.86,358.70){\usebox{\plotpoint}}

\put(716.29,368.23){\usebox{\plotpoint}}

\put(734.68,377.84){\usebox{\plotpoint}}

\put(753.41,386.78){\usebox{\plotpoint}}

\put(771.59,396.77){\usebox{\plotpoint}}

\put(790.09,406.18){\usebox{\plotpoint}}

\put(808.34,416.04){\usebox{\plotpoint}}

\put(827.05,425.03){\usebox{\plotpoint}}

\put(845.39,434.72){\usebox{\plotpoint}}

\put(863.76,444.38){\usebox{\plotpoint}}

\put(882.48,453.34){\usebox{\plotpoint}}

\put(900.70,463.27){\usebox{\plotpoint}}

\put(919.21,472.64){\usebox{\plotpoint}}

\put(937.59,482.29){\usebox{\plotpoint}}

\put(956.15,491.58){\usebox{\plotpoint}}

\put(974.52,501.23){\usebox{\plotpoint}}

\put(992.98,510.69){\usebox{\plotpoint}}

\put(1011.62,519.81){\usebox{\plotpoint}}

\put(1029.82,529.78){\usebox{\plotpoint}}

\put(1048.33,539.17){\usebox{\plotpoint}}

\put(1066.69,548.84){\usebox{\plotpoint}}

\put(1085.25,558.13){\usebox{\plotpoint}}

\put(1103.74,567.57){\usebox{\plotpoint}}

\put(1122.18,577.09){\usebox{\plotpoint}}

\put(1140.47,586.89){\usebox{\plotpoint}}

\put(1158.89,596.44){\usebox{\plotpoint}}

\put(1177.63,605.36){\usebox{\plotpoint}}

\put(1195.79,615.40){\usebox{\plotpoint}}

\put(1214.35,624.68){\usebox{\plotpoint}}

\put(1232.85,634.10){\usebox{\plotpoint}}

\put(1251.28,643.64){\usebox{\plotpoint}}

\put(1269.59,653.40){\usebox{\plotpoint}}

\put(1288.13,662.71){\usebox{\plotpoint}}

\put(1306.75,671.87){\usebox{\plotpoint}}

\put(1324.89,681.94){\usebox{\plotpoint}}

\put(1343.49,691.17){\usebox{\plotpoint}}

\put(1362.11,700.31){\usebox{\plotpoint}}

\put(1380.36,710.18){\usebox{\plotpoint}}

\put(1398.82,719.67){\usebox{\plotpoint}}

\put(1417.28,729.14){\usebox{\plotpoint}}

\put(1435.85,738.42){\usebox{\plotpoint}}

\put(1454.02,748.45){\usebox{\plotpoint}}

\put(1472.75,757.38){\usebox{\plotpoint}}

\put(1491.23,766.82){\usebox{\plotpoint}}

\put(1509.46,776.73){\usebox{\plotpoint}}

\put(1527.93,786.20){\usebox{\plotpoint}}

\put(1546.39,795.69){\usebox{\plotpoint}}

\put(1564.95,804.98){\usebox{\plotpoint}}

\put(1583.34,814.59){\usebox{\plotpoint}}

\put(1601.87,823.94){\usebox{\plotpoint}}

\put(1620.38,833.34){\usebox{\plotpoint}}

\put(1638.64,843.18){\usebox{\plotpoint}}

\put(1657.18,852.48){\usebox{\plotpoint}}

\put(1675.49,862.24){\usebox{\plotpoint}}

\put(1694.05,871.53){\usebox{\plotpoint}}

\put(1712.45,881.12){\usebox{\plotpoint}}

\put(1730.98,890.49){\usebox{\plotpoint}}

\put(1749.13,900.53){\usebox{\plotpoint}}

\put(1767.88,909.44){\usebox{\plotpoint}}

\put(1786.31,918.98){\usebox{\plotpoint}}

\put(1804.59,928.79){\usebox{\plotpoint}}

\put(1823.27,937.83){\usebox{\plotpoint}}

\put(1841.61,947.53){\usebox{\plotpoint}}

\put(1860.06,957.03){\usebox{\plotpoint}}

\put(1862,958){\usebox{\plotpoint}}

\put(251.0,131.0){\rule[-0.200pt]{0.400pt}{214.642pt}}

\put(251.0,131.0){\rule[-0.200pt]{410.975pt}{0.400pt}}

\put(1957.0,131.0){\rule[-0.200pt]{0.400pt}{214.642pt}}

\put(251.0,1022.0){\rule[-0.200pt]{410.975pt}{0.400pt}}

\end{picture}
    \end{center}
\caption{$f^2$ -- $m$ függvény}  \end{figure}

Az  egyenes meredeksége $a=\unitfrac[136,9]{Hz^2}{g}$, ebből \[\mu = \frac{n^2g}{4L^2a} = \unit[0,079]{\frac{g}{m}}.\]

Az eredményt ellenőrizendő, megmértük analitikai mérleggel a kötél egy $L_0 = 4,05$ méteres darabját, erre $m_0 =\unit[0,3925]{g}$-ot kaptunk. Így \[\mu_0 = \frac{m_0}{L_0}= \unit[0,0969]{\frac{g}{m}}.\]

\paragraph{Diszkusszió}

A két eredmény nagyságrendben megegyezik, mégis van a hibahatárnál nagyobb eltérés. Erre nem találtam kielégítő magyarázatot.





\end{document}
%% vim:lbr
