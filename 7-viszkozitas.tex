\documentclass[12pt]{article}
\usepackage[utf8]{inputenc}
\usepackage[final]{graphics}
\usepackage[pdftex]{graphicx}
\usepackage[dvips]{color}
\usepackage{amsfonts}
\usepackage{subfigure}
\usepackage{lscape}
\usepackage{hyperref}
\usepackage{amsmath}
\usepackage{units}
\usepackage{float}
\usepackage[table]{xcolor}
\usepackage{rmpage}
\usepackage[magyar]{babel}
\usepackage[margin=2cm]{geometry}
\newcommand{\hide}[1]{}
\DeclareGraphicsExtensions{.jpg,.pdf,.mps,.png}
\begin{document}
\def\rhog{\ensuremath{\rho_{g}}}\def\rhof{\ensuremath{\rho_{f}}}\def\rhov{\ensuremath{\rho_{v}}}\def\rho{\ensuremath{\varrho}}\def\dx{\ensuremath{d_x}}\def\dy{\ensuremath{d_y}}\def\dz{\ensuremath{d_z}}\def\mui{\ensuremath{\mu_1}}\def\muii{\ensuremath{\mu_2}}\def\muiii{\ensuremath{\mu_3}}\def\muiv{\ensuremath{\mu_4}}\def\tf{\ensuremath{t_f}}\def\Re{\ensuremath{Re}}\def\Deltaeta{\ensuremath{\Delta\eta}}
\paragraph{A mérés célja}
87\%-os glicerin belső súrlódását (viszkozitását) szeretnénk meghatározni.  A belső súrlódást így értelmezzük: ha egy folyadékban két párhuzamos, egymástól $d$ távolságra lévő, $A$ területű lemezt mozgatunk egymáshoz képest $v$ sebességgel, akkor a folyadék belső súrlódása miatt erő lép fel közöttük, ami egyenesen arányos $A$-val és $v$-vel, és fordítottan arányos $d$-vel. Az arányossági tényező, \[\eta = \frac{Fd}{Av} \] a viszkozitás.

\paragraph{A mérés leírása}
A viszkozitást kétféle módszerrel határozzuk meg. Először Höppler-féle viszkozi\-méterrel. Ehhez megmérjük, hogy a viszkoziméterben mennyi idő alatt ér le a golyó a két szélső osztás között ($t$), ill.\ megmérjük a folyadék sűrűségét ($\rhof$) areométerrel. Innen az \[\eta = K(\rhog-\rhof)t\] összefüggéssel kaphatjuk meg a viszkozitást, ahol $\rhog$ a viszkoziméter golyójának sűrűsége ($8,19\unitfrac{g}{cm^3}$), $K$ pedig a viszkoziméterre jellemző állandó. A glicerin hőmérsékletét is megmérjük ($\tf$), hogy össze tudjuk hasonlítani a vizskozitást az irodalmi értékkel.

Másodszor a Stokes törvény segítségével mérjük a viszkozitást.
Ha egy test mozog az álló folyadékhoz képest, akkor a sebességével ellentétes irányban közegellenállási erő hat rá, gömb esetén \[F=6\pi \eta r v.\]
Ha a golyó lefelé mozog, akkor ezen kívül még a felhajóerő és a nezézségi erő is fog rá hatni, és olyan sebességgel fog mozogni, hogy ez a három erő egyensúlyt tartson: \[mg - F_{fel} - F = 0\] \[ \frac{4r^3\pi}{3}\cdot \rhog g - \frac{4r^3\pi}{3} \cdot \rhof g - 6\pi \eta r v = 0\] 
\[\eta = \frac29 \frac{(\rhog - \rhof) r^2 g}{v}\]
Ez viszont csak akkor igaz, ha nem képződnek örvények az áramláskor. Ehhez pedig az kell, hogy a Reynolds-szám egy kritikus érték alatt legyen: \[\Re = \frac{\rhof r v}{\eta} < 0,1 \]
Tehát ezt is ellenőrizni fogjuk.

A golyók sűrűségét piknométerrel mérjük meg. Ez egy edény, amibe pontosan tudunk adott térfogatú folyadékot tölteni. Megmérjük az edény tömegét üresen ($\mui$), félig telerakva golyókkal ($\muii$), ugyanígy, csak teletöltve vízzel ($\muiii$), és csak vízzel teletöltve ($\muiv$). Ebből a golyók sűrűségét a \[\rhog = \rhov \frac{\muii - \mui}{\muiv - \mui - \muiii + \muii}\] képlettel kaphatjuk meg, ahol $\rhov$ a víz sűrűsége.


\paragraph{A mért adatok} $  $ \relax{}


  
  \begin{table}[H]
    \begin{center}
      \begin{tabular}{|
c|
c|
c|
c|
c|
}
        \hline
        
\ensuremath{\unit[\text{\ensuremath{t}}]{(s)}} & 
\ensuremath{\unit[\text{\ensuremath{\rho_{f}}}]{(\unitfrac{kg}{m^3})}} & \ensuremath{\unit[\text{\ensuremath{\rho_{g}}}]{(\unitfrac{kg}{m^3})}} & \ensuremath{\unit[\text{\ensuremath{K}}]{(\unitfrac{Pa\cdot m^3}{kg})}} & \ensuremath{\unit[\text{\ensuremath{t_f}}]{({}^\circ{}C)}}
\\
        \hline\hline
        
116
 & 1224
 & 8190
 & \ensuremath{1,3\cdot 10^{-7}}
 & 23
\\
        \hline
      \end{tabular}
      \caption{A Höppler-viszkoziméterrel és az areométerrel mért adatok}
      \label{tab:}
    \end{center}
  \end{table}
  
  \begin{table}[H]
    \begin{center}
      \begin{tabular}{|
c|
c|
c|
c|
c|
c|
}
        \hline
        
 & 
\ensuremath{\unit[\text{\ensuremath{d_x}}]{(m)}} & \ensuremath{\unit[\text{\ensuremath{d_y}}]{(m)}} & \ensuremath{\unit[\text{\ensuremath{d_z}}]{(m)}} & \ensuremath{\unit[\text{\ensuremath{t}}]{(s)}} & \ensuremath{\unit[\text{\ensuremath{s}}]{(m)}}
\\
        \hline\hline
        
1
 & \ensuremath{3,93\cdot 10^{-3}}
 & \ensuremath{3,98\cdot 10^{-3}}
 & \ensuremath{3,90\cdot 10^{-3}}
 & 3,58
 & 0,278
\\
        \hline
        
2
 & \ensuremath{2,58\cdot 10^{-3}}
 & \ensuremath{2,86\cdot 10^{-3}}
 & \ensuremath{2,85\cdot 10^{-3}}
 & 6,63
 & 0,278
\\
        \hline
        
3
 & \ensuremath{3,95\cdot 10^{-3}}
 & \ensuremath{3,97\cdot 10^{-3}}
 & \ensuremath{3,97\cdot 10^{-3}}
 & 3,52
 & 0,278
\\
        \hline
        
4
 & \ensuremath{3,94\cdot 10^{-3}}
 & \ensuremath{3,92\cdot 10^{-3}}
 & \ensuremath{3,88\cdot 10^{-3}}
 & 3,57
 & 0,278
\\
        \hline
        
5
 & \ensuremath{1,29\cdot 10^{-3}}
 & \ensuremath{1,30\cdot 10^{-3}}
 & \ensuremath{1,29\cdot 10^{-3}}
 & 27,95
 & 0,278
\\
        \hline
        
6
 & \ensuremath{1,28\cdot 10^{-3}}
 & \ensuremath{1,26\cdot 10^{-3}}
 & \ensuremath{1,27\cdot 10^{-3}}
 & 28,05
 & 0,278
\\
        \hline
        
7
 & \ensuremath{1,10\cdot 10^{-3}}
 & \ensuremath{1,09\cdot 10^{-3}}
 & \ensuremath{1,10\cdot 10^{-3}}
 & 38,18
 & 0,278
\\
        \hline
      \end{tabular}
      \caption{A Stokes-törvénnyel mért adatok}
      \label{tab:}
    \end{center}
  \end{table}
  
  \begin{table}[H]
    \begin{center}
      \begin{tabular}{|
c|
c|
c|
c|
}
        \hline
        
\ensuremath{\unit[\text{\ensuremath{\mu_1}}]{(kg)}} & 
\ensuremath{\unit[\text{\ensuremath{\mu_2}}]{(kg)}} & \ensuremath{\unit[\text{\ensuremath{\mu_3}}]{(kg)}} & \ensuremath{\unit[\text{\ensuremath{\mu_4}}]{(kg)}}
\\
        \hline\hline
        
0,03850
 & 0,07040
 & 0,12585
 & 0,09960
\\
        \hline
      \end{tabular}
      \caption{A piknométerrel mért adatok}
      \label{tab:}
    \end{center}
  \end{table}

\paragraph{Kiértékelés}

A viszkoziméteres mérés alapján a viszkozitás $\eta = 0,105\,\unit{Pa\cdot{}s}$.

A golyók méreteit átlagoljuk, kiszámoljuk a sebességeiket, és a Reynolds-számokat az előző kísérletből megkapott $\eta$-val:

  
  \begin{table}[H]
    \begin{center}
      \begin{tabular}{|
c|
c|
c|
c|
c|
}
        \hline
        
 & 
\ensuremath{\unit[\text{\ensuremath{d}}]{(m)}} & \ensuremath{\unit[\text{\ensuremath{r}}]{(m)}} & \ensuremath{\unit[\text{\ensuremath{v}}]{(\unitfrac{m}{s})}} & \ensuremath{Re}
\\
        \hline\hline
        
1
 & \ensuremath{3,94\cdot 10^{-3}}
 & \ensuremath{1,968\cdot 10^{-3}}
 & 0,0777
 & 1,8
\\
        \hline
        
2
 & \ensuremath{2,76\cdot 10^{-3}}
 & \ensuremath{1,382\cdot 10^{-3}}
 & 0,0419
 & 0,68
\\
        \hline
        
3
 & \ensuremath{3,96\cdot 10^{-3}}
 & \ensuremath{1,982\cdot 10^{-3}}
 & 0,0790
 & 1,8
\\
        \hline
        
4
 & \ensuremath{3,91\cdot 10^{-3}}
 & \ensuremath{1,957\cdot 10^{-3}}
 & 0,0779
 & 1,8
\\
        \hline
        
5
 & \ensuremath{1,29\cdot 10^{-3}}
 & \ensuremath{6,47\cdot 10^{-4}}
 & \ensuremath{9,95\cdot 10^{-3}}
 & 0,075
\\
        \hline
        
6
 & \ensuremath{1,27\cdot 10^{-3}}
 & \ensuremath{6,35\cdot 10^{-4}}
 & \ensuremath{9,91\cdot 10^{-3}}
 & 0,073
\\
        \hline
        
7
 & \ensuremath{1,10\cdot 10^{-3}}
 & \ensuremath{5,48\cdot 10^{-4}}
 & \ensuremath{7,28\cdot 10^{-3}}
 & 0,047
\\
        \hline
      \end{tabular}
      \caption{A Reynolds számok kiszámítása}
      \label{tab:}
    \end{center}
  \end{table}

Láthatjuk, hogy csak a kisebb méretű (5--7) golyókra teljesül a $\Re < 0,1$ feltétel. Tehát csak ezek eredményeit fogjuk figyelembe venni.

A golyók sűrűsége a piknométeres mérés alapján $\rhog=\unitfrac[5600 \pm 110]{kg}{m^3}$.
Ezek alapján már ki tudjuk számítani a Stokes-törvény alapján a glicerin viszkozitását:

  
  \begin{table}[H]
    \begin{center}
      \begin{tabular}{|
c|
c|
}
        \hline
        
 & 
\ensuremath{\unit[\text{\ensuremath{\eta}}]{(Pa\cdot{}s)}}
\\
        \hline\hline
        
1
 & 0,481
\\
        \hline
        
2
 & 0,439
\\
        \hline
        
3
 & 0,479
\\
        \hline
        
4
 & 0,474
\\
        \hline
        
5
 & 0,405
\\
        \hline
        
6
 & 0,392
\\
        \hline
        
7
 & 0,398
\\
        \hline
      \end{tabular}
      \caption{A viszkozitások}
      \label{tab:}
    \end{center}
  \end{table}

Az így kapott $\eta$-k közül az 5--7 golyókra számítom az átlagot, és a szórás háromszorosát veszem hibakorlátnak, így $\unit[(0,40 \pm 0,020)]{Pa\cdot s}$-t kapok.

\paragraph{Eredménytáblázat} $ $

  
  \begin{table}[H]
    \begin{center}
      \begin{tabular}{|
c|
c|
c|
c|
c|
c|
}
        \hline
        
~ & 
\multicolumn{2}{c|}{Viszkoziméter} & \multicolumn{2}{c|}{Stokes-törvény} & \multicolumn{1}{c|}{Irodalmi érték}
\\\cline{2-3}\cline{4-5}\cline{6-6}
        
 & 
\ensuremath{\eta} & \ensuremath{\Delta\eta} & \ensuremath{\eta} & \ensuremath{\Delta\eta} & \ensuremath{\eta}
\\
        \hline\hline
        

 & 0,105
 & \ensuremath{4,1\cdot 10^{-3}}
 & 0,40
 & 0,020
 & 0,117
\\
        \hline
      \end{tabular}
      \caption{Eredménytáblázat}
      \label{tab:}
    \end{center}
  \end{table}
Az irodalmi érték forrása:  \verb+http://www.met.reading.ac.uk/~sws04cdw/viscosity_calc.html+, 87.0 $m/m\%$-os glicerinre számolva $23^\circ$C-on.

\paragraph{Diszkusszió}

A viszkoziméterrel kapott eredmény a hibahatáron túl, de nem sokkal többel tér el az irodalmi értéktől. Ennek az eltérésnek lehet az az oka, hogy a glicerin nem pontosan a dobozon feltüntetett tömegszázalékos volt, hanem ennél hígabb. Erre enged következtetni az is, hogy a mért sűrűség ($\unitfrac[1224]{kg}{m^3}$) valamivel kisebb, mint amit fent említett webolalon kiszámolhatunk ($\unitfrac[1227]{kg}{m^3}$).

A Stokes-törvénnyel kapott eredmények viszont teljesen rosszak. Erre nem találtam jobb ma\-gyarázatot, mint hogy esetleg a kisebb golyók sűrűsége jóval kisebb a nagyobb golyókénál. A sűrűségméréskor ugyanis csak a nagyobb golyók sűrűségét vettük figyelembe. Így elképzelhető, hogy a kis (5--7) golyóknál ez okozza az eltérést, a nagy golyóknál pedig az örvényes áramlás, mivel ott a Reynolds-szám nem 0,1, hanem 1 körüli. De valószínűleg nem ez lesz a hiba oka, mert az örvényes áramlás miatt nem kéne ekkora eltérésnek mutatkozni.




\end{document}
